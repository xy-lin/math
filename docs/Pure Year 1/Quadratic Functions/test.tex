<p align="center">
<img src="/images/citizenkane1.webp" alt="drawing" width="500"/>
</p>

<div style="font-size: 22px;">

<br><br>
1. Find the set of values of $a$, where $a \neq 0$, for which the equation $a x^{2}+3 x+2=0$ has two distinct roots.

+++ <span style="color:green">Solutions</span>

<div style="font-size: 22px">

\begin{align*}
D & = b^{2}-4 a c \\
  & = 3^{2}-4 \times a \times 2 \\
  & = 9-8 a \\
\end{align*}
For two distinct roots, $D>0$.
\begin{align*}
9-8 a & >0 \\
9 & >8 a \\
a&< \frac{9}{8}\\
\end{align*}

</div>

+++

<br>

2. The graph shows the curve $y=a x^{2}+b x+c$.
![](https://cdn.mathpix.com/cropped/2025_10_25_9bb7d6eeebe0eb10bad4g-1.jpg?height=512&width=626&top_left_y=559&top_left_x=305)
Find the values of $a, b$ and $c$.

+++ <span style="color:blue">Hint</span>

<div style="font-size: 22px;">

Use the completed square form of a quadratic, and use the turn point and y-intercept to find all parameters.

</div>

+++

+++ <span style="color:green">Solutions</span>

<div style="font-size: 22px">

The turning point is at $(2,3)$, so the completed square form is:
\[y=a(x-2)^{2}+3\]
The y-intercept is at $(0, -5)$, so substituting these values in gives:
\begin{align*}
-5 & =a(0-2)^{2}+3 \\
-5 & =4 a+3 \\
-8 & =4 a \\
a & =-2 \\
\end{align*}
Substituting $a$ back into the completed square form gives:
\begin{align*}
y & =-2(x-2)^{2}+3 \\
    & =-2(x^{2}-4 x+4)+3 \\
    & =-2 x^{2}+8 x-8+3 \\
    & =-2 x^{2}+8 x-5 \\
\end{align*}
So, $a=-2$, $b=8$ and $c=-5$.

</div>

+++

<br>

3. The quadratic equation $2 x^{2}+5 x+k=0$ has equal roots.
(a) Find the value of $k$.


+++ <span style="color:green">Solutions</span>

<div style="font-size: 22px">

For equal roots, the discriminant $D=0$.
\begin{align*}
D & = b^{2}-4 a c \\
  & = 5^{2}-4 \times 2 \times k \\
  & = 25-8 k \\ 
0 & =25-8 k \\
8 k & =25 \\
k & =\frac{25}{8} \\
\end{align*}

</div>

+++

<br>

(b) Solve the equation $2 x^{2}+5 x+k=0$.

+++ <span style="color:green">Solutions</span>

<div style="font-size: 22px">

Substituting $k=\frac{25}{8}$ into the equation gives:
\begin{align*}
2 x^{2}+5 x+\frac{25}{8} & =0 \\
16 x^{2}+40 x+25 & =0 \\
(4 x+5)^{2} & =0 \\
4 x+5 & =0 \\
4 x & =-5 \\
x & =-\frac{5}{4} \\
\end{align*}

</div>

+++

<br>

4. In this question you must show detailed reasoning.
Show that the equation $3 x^{2}=2 x-4$ has no real roots.

+++ <span style="color:green">Solutions</span>

<div style="font-size: 22px">

Rearranging the equation gives:
\begin{align*}
3 x^{2}-2 x+4 & =0 \\
\end{align*}
Calculating the discriminant:
\begin{align*}
D & = b^{2}-4 a c \\
  & = (-2)^{2}-4 \times 3 \times 4 \\
  & = 4-48 \\
  & =-44 \\
\end{align*}
Since $D<0$, the equation has no real roots.


</div>

+++

<br>

5. Solve these equations, giving your answers in exact form.
(a) $x^{\frac{2}{3}}+x^{\frac{1}{3}}-6=0$

+++ <span style="color:green">Solutions</span>

<div style="font-size: 22px">


Let $y=x^{\frac{1}{3}}$. Then the equation becomes:
\begin{align*}
y^{2}+y-6 & =0 \\
(y+3)(y-2) & =0 \\
y+3 & =0 \quad \text{or} \quad y-2=0 \\
y & =-3 \quad \text{or} \quad y=2 \\
\end{align*}
Substituting back for $x$ gives:
\begin{align*}
x^{\frac{1}{3}} & =-3 \quad \text{or} \quad x^{\frac{1}{3}}=2 \\
x & =-27 \quad \text{or} \quad x=8 \\
\end{align*}


</div>

+++

<br>

(b) $x^{4}+3 x^{2}-10=0$

+++ <span style="color:green">Solutions</span>

<div style="font-size: 22px">

Let $y=x^{2}$. Then the equation becomes:
\begin{align*}
y^{2}+3 y-10 & =0 \\
(y+5)(y-2) & =0 \\
y+5 & =0 \quad \text{or} \quad y-2=0 \\
y & =-5 \quad \text{or} \quad y=2 \\
\end{align*}
Substituting back for $x$ gives:
\begin{align*}
x^{2} & =-5 \quad \text{or} \quad x^{2}=2 \\
\text{No real solution} & \quad \text{or} \quad x=\pm \sqrt{2} \\
\end{align*}

</div>

+++

<br>

6. (a) Express $2 x^{2}+p x+q$ in the form $a(x+b)^{2}+c$.

+++ <span style="color:green">Solutions</span>

<div style="font-size: 22px">

\begin{align*}
2 x^{2}+p x+q & =2\left(x^{2}+\frac{p}{2} x\right)+q \\
 & =2\left[x^{2}+\frac{p}{2} x+\left(\frac{p}{4}\right)^{2}-\left(\frac{p}{4}\right)^{2}\right]+q \\
 & =2\left(x+\frac{p}{4}\right)^{2}-2\left(\frac{p}{4}\right)^{2}+q \\
 & =2\left(x+\frac{p}{4}\right)^{2}-\frac{p^{2}}{8}+q \\
\end{align*}

</div>

+++

<br>

(b) The curve $y=2 x^{2}+p x+q$ has a minimum point at $(3,-2)$.
Find the values of $p$ and $q$.

+++ <span style="color:green">Solutions</span>

<div style="font-size: 22px">

Use the form obtained in question a and the turning point to find $p$ and $q$.

We have:
$$
2\left(x+\frac{p}{4}\right)^{2}-\frac{p^{2}}{8}+q
$$

At the minimum point, $x=3$ and $y=-2$. Substituting these values in gives:

\begin{align*}
\frac{p}{4} &=-3\\
-\frac{p^{2}}{8}+q &=-2\\
\end{align*}

solve above:
\begin{align*}
p & =-12 \\
\end{align*}

\begin{align*}
-\frac{(-12)^{2}}{8}+q & =-2 \\
-\frac{144}{8}+q & =-2 \\
-18+q & =-2 \\
q & =16 \\
\end{align*}

</div>

+++

<br>

7. The diagram shows a right-angled triangle. Find the value of $x$, correct to 3 s.f.
<p align="center">
<img src="/assets/Pure1_Quadratic functions Q7.png" alt="drawing" width="500"/>
</p>

+++ <span style="color:green">Solutions</span>

<div style="font-size: 22px">

Using Pythagoras' theorem:
\begin{align*}
(2x+1)^{2} & =x^2+3^2 \\
4 x^{2}+4 x+1 & =x^{2}+9 \\
3 x^{2}+4 x-8 & =0 \\
\end{align*}

Using the quadratic formula:
\begin{align*}
x & =\frac{-4 \pm \sqrt{4^{2}-4 \times 3 \times (-8)}}{2 \times 3} \\
  & =\frac{-4 \pm \sqrt{16+96}}{6} \\
  & =\frac{-4 \pm \sqrt{112}}{6} \\
  & =\frac{-4 \pm 4 \sqrt{7}}{6} \\
  & =\frac{-2 \pm 2 \sqrt{7}}{3} \\ 
\end{align*}
Since $x$ must be positive:
\begin{align*}
x & =\frac{-2 + 2 \sqrt{7}}{3} \\
 & \approx 1.65 \\
\end{align*}

</div>

+++

<br>

8. Amy throws a ball so that when it is at its highest point, it passes through the centre of a hoop. The path of the ball is modelled by the equation $y=h+k x-\frac{1}{2} x^{2}$, where $y$ is the height of the ball in metres above the ground and $x$ is the horizontal distance in metres from the point at which the ball was thrown. The centre of the hoop is at the point where $x=2$ and $y=5$.
(a) Find the values of $h$ and $k$.


+++ <span style="color:blue">Hint</span>

<div style="font-size: 22px;">

Use the completed square form of a quadratic and turning point to find $k$ and then substitute to find $h$.
The turning point is at $(2,5)$.

</div>

+++

+++ <span style="color:green">Solutions</span>

<div style="font-size: 22px">

Use the completed square form to find $k$ since the highest point is at $x=2, y=5$.
\begin{align*}
y & =\frac{-1}{2}(x-p)^{2}+q \\
y & =\frac{-1}{2}(x-2)^{2}+5 \\
    & =\frac{-1}{2}(x^{2}-4 x+4)+5 \\
    & =\frac{-1}{2} x^{2}+2 x-2+5 \\
    & =\frac{-1}{2} x^{2}+2 x+3 \\
\end{align*}
So, $h=3$ and $k=2$.

</div>

+++

<br>

(b) Find the value of $x$ at which the ball hits the ground.


+++ <span style="color:blue">Hint</span>

<div style="font-size: 22px;">

When the ball hits the ground, $y=0$. Set $y=0$ and solve for $x$.

</div>

+++

+++ <span style="color:green">Solutions</span>

<div style="font-size: 22px">

To find when the ball hits the ground, set $y=0$ and solve for $x$:
\begin{align*}
0 & =3+2 x-\frac{1}{2} x^{2} \\
0 & =-6-4 x+x^{2} \\
0 & =x^{2}-4 x-6 \\
\end{align*}
Using the quadratic formula:
\begin{align*}
x & =\frac{4 \pm \sqrt{(-4)^{2}-4 \times 1 \times (-6)}}{2 \times 1} \\
  & =\frac{4 \pm \sqrt{16+24}}{2} \\
  & =\frac{4 \pm \sqrt{40}}{2} \\
  & =\frac{4 \pm 2 \sqrt{10}}{2} \\
  & =2 \pm \sqrt{10} \\
\end{align*}
Since $x$ must be positive:
\begin{align*}
x & =2 + \sqrt{10} \\
 & \approx 5.16 \\
\end{align*}

</div>

+++

<br>
<br>

</div>
<p align="center">
<img src="/images/citizenkane2.webp" alt="drawing" width="500"/>
</p>
