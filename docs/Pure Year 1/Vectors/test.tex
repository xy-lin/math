<p align="center">
<img src="/images/" alt="drawing" width="500"/>
</p>

<div style="font-size: 22px;">

<br><br>
1. Given that $\mathbf{p}=\binom{1}{4}$ and $\mathbf{q}=\binom{3}{-1}$
(a) Find $2 \mathbf{p}+\mathbf{q}$.
(b) The vector $\mathbf{p}+k \mathbf{q}$ is parallel to the vector $\binom{1}{1}$. Find $k$.
\((c)\) Find $a$ and $b$ such that $a \mathbf{p}+b \mathbf{q}=\binom{1}{0}$.
2. The points $\mathrm{A}, \mathrm{B}$ and C have coordinates $(-4,0),(2,-1)$ and $(3,2)$ respectively.
(a) Write down the vectors $\overrightarrow{\mathrm{AB}}, \overrightarrow{\mathrm{AC}}$ and $\overrightarrow{\mathrm{BC}}$.
(b) Write down an equation linking vectors $\overrightarrow{\mathrm{AB}}, \overrightarrow{\mathrm{AC}}$ and $\overrightarrow{\mathrm{BC}}$.
\((c)\) Find a unit vector in the direction of $\overrightarrow{\mathrm{BC}}$.
(d) A fourth point D is positioned so that ABCD is a trapezium with AD parallel to BC and $|\overrightarrow{\mathrm{AD}}|=1$.
![](https://cdn.mathpix.com/cropped/2025_10_25_f89e9b416619032c13e8g-1.jpg?height=310&width=600&top_left_y=1225&top_left_x=444)

Find the coordinates of D .
3. OABC is a quadrilateral. Relative to point O , points $\mathrm{A}, \mathrm{B}$ and C have position vectors $\mathbf{a , b , c}$ respectively. The midpoints of $\mathrm{AB}, \mathrm{BC}, \mathrm{CO}$ and OA are the points $\mathrm{P}, \mathrm{Q}, \mathrm{R}$ and S respectively.
(a) Show P has position vector $\frac{1}{2}(\mathbf{a}+\mathbf{b})$
(b) Hence write down the position vectors of $\mathrm{Q}, \mathrm{R}$ and S .
\((c)\) Show that $\overrightarrow{\mathrm{PQ}}=\overrightarrow{\mathrm{SR}}$.
(d) State, with a reason, what geometrical shape PQRS is.
4. Relative to the point O , points $\mathrm{A}, \mathrm{B}$ and C have position vectors $\binom{a}{1},\binom{-3}{6}$ and $\binom{7}{4}$ respectively. M is the midpoint of BC .

(a) You are given that $|\mathrm{OA}|+|\mathrm{AM}|=|\mathrm{OM}|$. Explain why O , A and M must form a straight line.
(b) Hence find the value of $a$.
5. The corners of triangle PQR lie on a circle. $\overrightarrow{\mathrm{PQ}}=3 \mathbf{i}+4 \mathbf{j}$ and $\overrightarrow{\mathrm{PR}}=-8 \mathbf{i}+6 \mathbf{j}$.
(a) Find the lengths of $\mathrm{PQ}, \mathrm{QR}$ and PR and explain how this shows that the triangle is rightangled.
(b) Hence find the area of the circle.

Total 40 marks
<br>

</div>
<p align="center">
<img src="/images/" alt="drawing" width="500"/>
</p>
