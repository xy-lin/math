<p align="center">
<img src="/images/" alt="drawing" width="500"/>
</p>

<div style="font-size: 22px;">

<br><br>

Use $g=9.8 m s^{-2}$ unless stated otherwise.
1. A particle is moving with constant velocity. Three forces $\mathbf{F}_{1}, \mathbf{F}_{2}$ and $\mathbf{F}_{3}$ act on the particle, where $\mathbf{F}_{1}=\binom{4}{-3}, \mathbf{F}_{2}=\binom{-6}{p}$ and $\mathbf{F}_{3}=\binom{q}{0}$. Find the values of $p$ and $q$.
2. (a) Calculate the acceleration of an object of mass 150 kg subject to a net force of 60 N .
(b) A load of mass 150 kg is accelerating vertically upwards as the result of the pull of a crane wire. The tension in the wire is 1488 N .
Assuming that the only forces acting on the load are its weight and the tension in the wire, calculate the acceleration of the load.
3. A car of mass 1500 kg is pulling a trailer of mass 900 kg along a straight, horizontal road. The coupling between the car and the trailer is light, rigid and horizontal.

The motion of the car and trailer is modelled assuming that the resistances to motion are negligible. There is a driving force of 600 N acting on the car.
(a) Draw separate diagrams showing the horizontal force(s) acting on
(i) the car.
(ii) the trailer.
(b) Calculate the acceleration of the car and trailer.

The situation is remodelled to include a constant resistant force of 150 N on the car and 100 N on the trailer.
\((c)\) Find the tension in the coupling.
4. In this question, the unit vector $\mathbf{i}$ is horizontal and the unit vector $\mathbf{j}$ is vertically upwards. All forces are in Newtons.

A small, heavy box is suspended in mid-air and is held in equilibrium by the tension in a light inextensible string and a horizontal force, as shown in the diagram.
![](https://cdn.mathpix.com/cropped/2025_10_25_8aa54799b8ab9dbe56bdg-1.jpg?height=315&width=432&top_left_y=2110&top_left_x=502)
![](https://cdn.mathpix.com/cropped/2025_10_25_8aa54799b8ab9dbe56bdg-1.jpg?height=212&width=204&top_left_y=2110&top_left_x=1176)

The tension in the string is $\mathbf{T}_{1}$, where $\mathbf{T}_{1}=30 \mathbf{i}+49 \mathbf{j}$. The horizontal force is $\mathbf{F}_{1}$, where $\mathbf{F}_{1}=p \mathbf{i}$.
(a) Find the value of $p$.

(b) Find the mass $m$.
\((c)\) Calculate the magnitude of $\mathbf{T}_{1}$ and the angle that $\mathbf{T}_{1}$ makes with the horizontal.

Another force $\mathbf{F}_{2}=48 \mathbf{i}-87 \mathbf{j}$ is now applied to the box. The force $\mathbf{F}_{1}$ still acts and the box is still in equilibrium.
(d) The new tension in the string is $\mathbf{T}_{2}=a \mathbf{i}+b \mathbf{j}$. Calculate the values of $a$ and $b$.
5. A girl of mass 48 kg takes a lift from the ground floor up to the second floor. She is holding a package that weighs 5 kg by means of a light inextensible string.
The lift initially accelerates at $2 \mathrm{~ms}^{-2}$ and then travels at a constant speed of $5 \mathrm{~ms}^{-1}$. Finally, the lift decelerates at $3 \mathrm{~ms}^{-2}$.
The normal reaction of the floor of the lift on the girl is $R \mathrm{~N}$.
(a) Find the minimum value of $R$ during the motion.
(b) Given that the string does not break during the motion, find the maximum tension in the string during the motion.
6. Three particles, $\mathrm{A}, \mathrm{B}$ and C , are attached by light inextensible strings AB and BC . Particles A and B are held at an equal height of 2 m above a horizontal floor. B rests on a rough horizontal surface and when in motion experiences a constant resistive force of $P$ N. The string passes over two smooth pulleys.
![](https://cdn.mathpix.com/cropped/2025_10_25_8aa54799b8ab9dbe56bdg-2.jpg?height=521&width=898&top_left_y=1313&top_left_x=493)

The masses of the particles are $5.2 \mathrm{~kg}, 8 \mathrm{~kg}$ and 3.5 kg respectively.
It is given that B does not reach either pulley during any part of the motion following release.
The particles are released and A hits the ground 4 seconds later.
(a) Find the tension in each of the strings AB and BC .
(b) Find the exact value of $P$.
\((c)\) Explain how you have used the modelling assumption that the pulleys are smooth. [1] The particle A detaches from the string as soon as it hits the floor.
(d) Find the acceleration of the system when C is moving downwards.
<br>

</div>
<p align="center">
<img src="/images/" alt="drawing" width="500"/>
</p>
