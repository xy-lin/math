<p align="center">
<img src="/images/wallE1.webp" alt="drawing" width="500"/>
</p>

<div style="font-size: 22px;">

<br><br>

Use $g=9.8 m s^{-2}$ unless stated otherwise.
1. A particle traveling in a straight line at $15 \mathrm{~ms}^{-1}$ is brought to rest by a constant deceleration in a distance of 22.5 m . Find the time taken for the particle to come to rest.
2. A plant pot falls from a balcony. It hits the ground after 2.24 seconds.
(a) By modelling the plant pot as a particle, find the height of the balcony from the ground.
(b) State one further assumption used to model this situation.
3. A particle accelerates uniformly from $7 \mathrm{~ms}^{-1}$ to $21 \mathrm{~ms}^{-1}$ in 8 s . How far does it travel in this time?
4. A particle is projected at a speed of $12 \mathrm{~ms}^{-1}$ up a straight inclined track. Whilst on the track the particle experiences a constant acceleration down the track of $5 \mathrm{~ms}^{-2}$. Find the length of time that the displacement of the particle up the track from the point of projection is greater than 8 metres.
5. A train passes through station A tramelling at $10 \mathrm{~ms}^{-1}$ along a straight track.

For the first 5 seconds after leaving the station the train has a constant acceleration of $7 \mathrm{~ms}^{-2}$.

The train then travels with constant velocity for a time, T , until it decelerates uniformly for 15 seconds, coming to rest at the station $B$ which is 2.68 km away from $A$.

Sketch a velocity time graph for the motion of the train and calculate the time taken to get between the two stations.
6. A particle travels in a straight line.

The motion is modelled by the $v$ - $t$ diagram below.
![](https://cdn.mathpix.com/cropped/2025_10_25_2b1401b6ed12b43ff6ecg-1.jpg?height=480&width=1152&top_left_y=1936&top_left_x=319)
(a) Calculate the acceleration of the particle in the part of the motion from $t=1$ to $t=4$.
(b) Calculate the displacement of the particle from its position when $t=0$ to its position when $t=7$.
7. A ball A is thrown vertically upwards at $25 \mathrm{~ms}^{-1}$ from a point P . Three seconds later a second ball $B$ is also thrown vertically upwards from the point $P$ at $25 \mathrm{~ms}^{-1}$. Taking the acceleration due to gravity to be $10 \mathrm{~ms}^{-2}$, calculate
(a) the time for which ball A has been in motion when the balls meet
(b) the height above P at which A and B meet.
8. Cars A and B are traveling in the same direction along a straight road. The time $t$ is in seconds.

At $t=0$, car A is at rest. It accelerates at $3 \mathrm{~ms}^{-2}$ for $0 \leq t \leq 10$ and then travels at a constant speed.

Car B travels at $15 \mathrm{~ms}^{-1}$ for $0 \leq t \leq 30$ and then accelerates at $1 \mathrm{~ms}^{-2}$ until it reaches a speed of $25 \mathrm{~ms}^{-1}$, after which it continues at this constant speed.
(a) Draw $v$ - $t$ diagrams for the motion of car A and of car B , where $v$ is the speed in $\mathrm{ms}^{-1}$ and $0 \leq t \leq 80$.
(b) Show that, in the first 40 seconds, car A travels 400 m further than car B.
\((c)\) Given that car A is 500 m behind car B at $t=0$, at what value of $t$ does car A catch up with car B ?
<br>

</div>
<p align="center">
<img src="/images/wallE2.avif" alt="drawing" width="500"/>
</p>
