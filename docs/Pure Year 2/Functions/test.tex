<p align="center">
<img src="/images/godFather.jpg" alt="drawing" width="500"/>
</p>

<div style="font-size: 22px;">

<br><br>

1. The functions g and h are defined as follows:
$$
\begin{array}{ll}
\mathrm{g}(x)=\sqrt{x} & x \geq 0 \\
\mathrm{~h}(x)=2 x+1 & x \in \mathbb{R}
\end{array}
$$

Find the following functions, giving the domain and range of each.
$$
\begin{aligned}
& \text { (a) } \operatorname{hg}(x) \\
& \text { (b) } \operatorname{gh}(x)
\end{aligned}
$$
2. (a) Sketch the graph of $y=|2 x+1|$.
(b) Hence, or otherwise, solve each of the following equations:
(i) $\quad|2 x+1|=3-x$
(ii) $\quad|2 x+1|=3 x-2$
3. The diagram below shows the graph $y=\mathrm{f}(x)$, where $\mathrm{f}(x)=\frac{x-1}{x}$ for $x>0$.

The graph approaches the line $y=1$ as $x$ becomes very large.
![](https://cdn.mathpix.com/cropped/2025_10_25_ff1d386d381ea0ca30a6g-1.jpg?height=384&width=549&top_left_y=1324&top_left_x=599)
(a) Write down the domain and range of $\mathrm{f}(x)$.
(b) Find the inverse function $\mathrm{f}^{-1}(x)$.
(c) Write down the domain and range of $\mathrm{f}^{-1}(x)$.
(d) Sketch the graph of $y=\mathrm{f}^{-1}(x)$ for the domain you gave in (iii).
(e) What is the relationship between the graph of $y=\mathrm{f}(x)$ and the graph of
$$
y=\mathrm{f}^{-1}(x) ?
$$
4. A biologist observes a population of rabbits regularly over a period of several years. Her observations show that for the first 2 years the population size doubles approximately every six months, from an initial population of 50 rabbits, before levelling off. She suggests the following model for the population in the first 2 years:
$$
\mathrm{f}(t)=A\left(2^{k t}\right), \quad 0 \leq t \leq 2
$$
where $\mathrm{f}(t)$ is the number of rabbits, $t$ is the time in years and $A$ and $k$ are constants.
(a) Give values for $A$ and $k$, explaining your reasoning carefully.

The biologist is interested in how much food is required to sustain the rabbit population. She suggests a model of:
$$
\mathrm{g}(x)=c x, \quad x>0
$$
where $\mathrm{g}(x)$ is the number of units of food consumed daily by the population, $x$ is the number of rabbits in the population and $c$ is a constant.
(b) Explain what the coefficient of $x$ represents in this model.
(c) State one assumption made by the biologist.
(d) Explain why the given domain for the function is unlikely to be a good model.
(e) Find the composite function $\operatorname{gf}(t)$ in terms of $c$, stating its domain clearly.

Explain what relationship is given by this composite function.
(f) Explain why it does not make sense to find the composite function $\mathrm{fg}(x)$.
5. The graph of a function $y=\mathrm{f}(x)$ is shown below. The graph has a local maximum at $(-1,1)$ and a local minimum at $(2,-2)$.
![](https://cdn.mathpix.com/cropped/2025_10_25_ff1d386d381ea0ca30a6g-2.jpg?height=557&width=720&top_left_y=965&top_left_x=608)

Sketch each of the following graphs, giving the coordinates of the turning points in each case.
(a) $y=\mathrm{f}(2 x)-1$
(b) $y=2 \mathrm{f}(x-1)$
(c) $y=\mathrm{f}(2 x-1)$
(d) $y=2 \mathrm{f}(x)-1$
<br>

</div>
<p align="center">
<img src="/images/godFather2.jpg" alt="drawing" width="500"/>
</p>
