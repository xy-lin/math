<p align="center">
<img src="/images/fightClub.jpeg" alt="drawing" width="500"/>
</p>

<div style="font-size: 22px;">

<br><br>

1. Write $\frac{9}{(1-x)(1+2 x)^{2}}$ as partial fractions.
2. (a) Expand $(3-x)^{-4}$ in ascending powers of $x$ up to and including the term in $x^{3}$, stating the range for which the expansion is valid.
(b) Use this expansion to calculate $(2.99)^{-4}$ correct to 4 decimal places.
3. The function f is given by
$$
f(x)=\frac{6}{1-9 x^{2}}
$$
(a) Write $\mathrm{f}(x)$ as a sum of two partial fractions.
(b) Expand your fractions to find a quadratic approximation to $\mathrm{f}(x)$.
(c) Explain why your expansion will not give a valid approximation to $\mathrm{f}(0.4)$.
4. Given that $\mathrm{f}(x)=\frac{x^{3}-2 x^{2}+3}{x+3}=A x^{2}+B x+C+\frac{D}{x+3}$, find the values of $A, B, C$ and $D$.
5. (a) Expand $(1+3 x)^{\frac{1}{2}}$ up to and including the term in $x^{3}$.
(b) Mo uses the expansion from part (a) and the value $x=2$ to find a value for $\sqrt{7}$. Find Mo's result and explain why this is not a valid approximation.
(c) Use the expansion from part (a) to find a value for $\sqrt{1.75}$ to 2 decimal places and explain why this is a valid approximation.
(d) Use your result from (c) to find a value for $\sqrt{7}$ correct to 2 decimal places.
6. $\frac{1+x}{1-2 x}$ is approximately equal to $1+a x+b x^{2}$
(a) Find the values of $a$ and $b$.
(b) Emily wants to use this expansion to calculate an approximation for $\int_{0}^{1} \frac{1+x}{1-2 x} \mathrm{~d} x$.

Give one reason why using this expansion would not give a valid approximation.
7. (a) Write $\mathrm{f}(x)=\frac{4}{(x-1)(x+3)}$ as partial fractions.
(b) Find the $x$ coordinate(s) of any turning points on the curve $y=\mathrm{f}(x)$ and identify their nature.
<br>

</div>
<p align="center">
<img src="/images/fightClub2.jpeg" alt="drawing" width="500"/>
</p>
