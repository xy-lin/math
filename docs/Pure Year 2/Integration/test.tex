<p align="center">
<img src="/images/starwar.png" alt="drawing" width="500"/>
</p>

<div style="font-size: 22px;">

<br><br>

1. The diagram below shows the graphs of $y=x^{2}+x+1$ and $y=5+3 x-x^{2}$.
![](https://cdn.mathpix.com/cropped/2025_10_22_7e064fd0c5eef051ee6ag-1.jpg?height=321&width=458&top_left_y=442&top_left_x=511)

Calculate the shaded area.
<br>

+++ <span style="color:blue">Hint</span>

<hr style="border:1px solid red" >

1. Find the points of intersection of the curves by solving the equation $x^{2}+x+1=5+3 x-x^{2}$.
2. Find the area between the curves by evaluating the integral of the difference of the functions between the points of intersection.

<hr style="border:1px solid red" >

+++

<br>

+++ <span style="color:green">Solutions</span>

<hr style="border:1px solid red" >

1. The points of intersection are found by solving the equation:
\[x^{2}+x+1=5+3 x-x^{2}\]
Rearranging gives:
\[2 x^{2}-2 x-4=0\]

Solve above, we get $x=-1$ and $x=2$.

<br>

2. The area between the curves is given by the integral:
\[\int_{-1}^{2}[(5+3 x-x^{2})-(x^{2}+x+1)] \mathrm{d} x\]
Simplifying the integrand:
\[\int_{-1}^{2}(-2 x^{2}+2 x+4) \mathrm{d} x\]
Calculating the integral:
\[\left[-\frac{2}{3} x^{3}+x^{2}+4 x\right]_{-1}^{2}\]
Evaluating at the limits:
\[\left(-\frac{16}{3}+4+8\right)-\left(\frac{2}{3}+1-4\right)=\frac{27}{3}=9\]

<hr style="border:1px solid red" >

+++

<br>

2. Find the following indefinite integrals, using any appropriate method.
(a) $\int \frac{x^{2}}{\left(x^{3}+2\right)^{2}} \mathrm{~d} x$

+++ <span style="color:blue">Hint</span>

<hr style="border:1px solid red" >

Use the substitution $u=x^{3}+2$.

<hr style="border:1px solid red" >

+++

<br>

+++ <span style="color:green">Solutions</span>

<hr style="border:1px solid red" >

Let $u=x^{3}+2$, then $\frac{\mathrm{d}{u}}{\mathrm{~d} x}=3 x^{2}$ $\rightarrow$ $\mathrm{d} x=\frac{\mathrm{d} u}{3 x^{2}}$. substitution into the original integral gives:
\[\int \frac{x^{2}}{u^{2}} \cdot \frac{\mathrm{d} u}{3 x^{2}}=\frac{1}{3} \int u^{-2} \mathrm{~d} u\]
Calculating the integral:
\[\frac{1}{3}\left[-u^{-1}\right]+C\]
substitution back to $x$ gives the final answer:
\[-\frac{1}{3\left(x^{3}+2\right)}+C\]
where $C$ is the constant of integration.

<hr style="border:1px solid red" >

+++

<br>

(b) $\int \frac{\mathrm{e}^{x}}{1+\mathrm{e}^{x}} \mathrm{~d} x$

<br>

+++ <span style="color:blue">Hint</span>

<hr style="border:1px solid red" >

The special case of integral by recognition: the reciprocal function: $\int \frac{f^{\prime}(x)}{f(x)} \mathrm{d} x=\ln |f(x)|+C$.

<hr style="border:1px solid red" >

+++

<br>

+++ <span style="color:green">Solutions</span>

<hr style="border:1px solid red" >

Use the reciprocal function rule, let $f(x)=1+e^x$, then $f'(x)=e^x$, so we have
so
$$
\int \frac{e^x}{1+e^x},dx=\ln\big(1+e^x\big)+C.
$$

<br>

If you prefer substitution: let $u=1+e^x$, then $du=e^x dx$, so the integral becomes $\int \frac{1}{u} du=\ln|u|+C=\ln(1+e^x)+C$.

<hr style="border:1px solid red" >

+++

<br>

3. In this question you must show detailed reasoning
Find the exact values of the following:
(a) $\int_{0}^{2} x \mathrm{e}^{x^{2}} \mathrm{~d} x$.

<br>

+++ <span style="color:blue">Hint</span>

<hr style="border:1px solid red" >

Integration by substitution, Let $u=x^{2}$

<hr style="border:1px solid red" >

+++

<br>

+++ <span style="color:green">Solutions</span>

<hr style="border:1px solid red" >

Let $u=x^{2}$, hence $du=2x\,dx$ and $x\,dx=\tfrac{1}{2}\,du$. Then

$$
\int_{0}^{2} x e^{x^{2}}\,dx
= \frac{1}{2}\int_{0}^{4} e^{u}\,du
= \frac{1}{2}\big[e^{u}\big]_{0}^{4}
= \frac{1}{2}\big(e^{4}-1\big).
$$

Therefore the exact value is

$$
\displaystyle \frac{e^{4}-1}{2}
$$

<hr style="border:1px solid red" >

+++

<br>

(b) $\int_{0}^{\pi / 2} \frac{\cos x}{\sin x+1} \mathrm{~d} x$.

<br>

+++ <span style="color:blue">Hint</span>

<hr style="border:1px solid red" >

This can use reciprocal rule or substitution

<hr style="border:1px solid red" >

+++

<br>

+++ <span style="color:green">Solutions</span>

<hr style="border:1px solid red" >
use substitution method:
Let $u=\sin x+1$, so $du=\cos x\,dx$. When $x=0$, $u=1$; when $x=\tfrac{\pi}{2}$, $u=2$. Thus

$$
\int_{0}^{\pi/2}\frac{\cos x}{\sin x+1}\,dx
= \int_{1}^{2}\frac{1}{u}\,du
= \big[\ln u\big]_{1}^{2}
= \ln 2.
$$

Therefore,

$$
\ln 2
$$

use reciprocal rule:
Let $f(x)=\sin x+1$, then $f'(x)=\cos x$, so we have
$$
\int \frac{\cos x}{\sin x+1},dx=\ln|\sin x+1|+C.
$$
Evaluating from $0$ to $\tfrac{\pi}{2}$ gives
$$
\big[\ln|\sin x+1|\big]_{0}^{\pi/2}=\ln 2-\ln 1=\ln 2.
$$

<hr style="border:1px solid red" >

+++

<br>

4. Find $\int x \sin 3 x \mathrm{~d} x$.

+++ <span style="color:blue">Hint</span>

<hr style="border:1px solid red" >

use "integration by parts"

<hr style="border:1px solid red" >

+++

<br>

+++ <span style="color:green">Solutions</span>

<hr style="border:1px solid red" >

Let $u=x$ and $dv=\sin 3x\,dx$. Then $du=dx$ and $v=-\tfrac{1}{3}\cos 3x$. By integration by parts:

$$
\begin{align}
\int x\sin 3x dx &= uv - \int v du\\
&= -\frac{x}{3}\cos 3x + \frac{1}{3}\int \cos 3x dx\\
&= -\frac{x}{3}\cos 3x + \frac{1}{9}\sin 3x + C\\
\end{align}
$$

Therefore,

$$
-\frac{x}{3}\cos 3x + \frac{1}{9}\sin 3x + C
$$

<hr style="border:1px solid red" >

+++

<br>


5. Use a substitution to show that $\int x \sqrt{1-x} \mathrm{~d} x=(A x+B)(\sqrt{1-x})^{3}+c$ where $A$ and $B$ are rational numbers to be found.

<br>

+++ <span style="color:blue">Hint</span>

<hr style="border:1px solid red" >

Let  $u=\sqrt{1-x}$ Then $1-x=u^{2}$, so $x=1-u^{2}$ and $dx=-2u\,du$
Alternatively, we can write $u=1-x$, but both methods lead to the same answer.

<hr style="border:1px solid red" >

+++

<br>

+++ <span style="color:green">Solutions</span>

<hr style="border:1px solid red" >

Compute the integral:

$$
\int x\sqrt{1-x}\,dx.
$$

Let  $u=\sqrt{1-x}$. Then $1-x=u^{2}$, so $x=1-u^{2}$ and $dx=-2u\,du$. Hence

$$
\int x\sqrt{1-x}\,dx
= \int (1-u^{2})\cdot u\cdot(-2u)\,du
= -2\int (u^{2}-u^{4})\,du.
$$

Integrating gives

$$
-2\left(\frac{u^{3}}{3}-\frac{u^{5}}{5}\right)
= -\frac{2}{3}u^{3}+\frac{2}{5}u^{5}+C.
$$

Factor $u^{3}$ and substitute $u^{2}=1-x$:

$$
-\frac{2}{3}u^{3}+\frac{2}{5}u^{3}(1-x)
= u^{3}\!\left(-\frac{2}{5}x-\frac{4}{15}\right).
$$

Therefore

$$
\int x\sqrt{1-x}\,dx=(A x+B)\big(\sqrt{1-x}\big)^{3}+C
$$

with

$$
A=-\frac{2}{5},\qquad B=-\frac{4}{15}.
$$

<br>

Alternatively, we can write $u=1-x$, 
$$
\int x\sqrt{1-x}\,dx.
$$


Let $u=1-x$. Then $du=-dx$, so $dx=-du$, and $x=1-u$, while $\sqrt{1-x}=\sqrt{u}=u^{1/2}$. Hence

$$
\int x\sqrt{1-x}\,dx
= \int (1-u)u^{1/2}(-du)
= -\int (1-u)u^{1/2}\,du
= -\int\big(u^{1/2}-u^{3/2}\big)\,du.
$$

Integrating gives

$$
-\left(\frac{2}{3}u^{3/2}-\frac{2}{5}u^{5/2}\right)
= -\frac{2}{3}u^{3/2}+\frac{2}{5}u^{5/2}+C.
$$

Factor $u^{3/2}$ and substitute back $u=1-x$:

$$
-\frac{2}{3}u^{3/2}+\frac{2}{5}u^{5/2}
= u^{3/2}\!\left(-\frac{2}{3}+\frac{2}{5}u\right)
= u^{3/2}\!\left(-\frac{2}{5}x-\frac{4}{15}\right).
$$

Therefore

$$
\int x\sqrt{1-x}\,dx=(A x+B)\big(\sqrt{1-x}\big)^{3}+C
$$

with

$$
A=-\frac{2}{5},\qquad B=-\frac{4}{15}.
$$

<hr style="border:1px solid red" >

+++

<br>

6. The diagram shows part of the curve $y=x^{2} \ln x$ for $x \geq 0$.
![](https://cdn.mathpix.com/cropped/2025_10_22_7e064fd0c5eef051ee6ag-1.jpg?height=512&width=632&top_left_y=1930&top_left_x=308)

R is the region above the $x$-axis bounded by the curve, the $x$ axis and the line $x=\mathrm{e}$.
Show that the area of R is $p \mathrm{e}^{3}+q$ where $p$ and $q$ are rational numbers to be found.

<br>

+++ <span style="color:blue">Hint</span>

<hr style="border:1px solid red" >

Find the point where the curve meets the x axis: solve $x^{2}\ln x=0$, since x is larger then 0, we have $\ln x=0$, so $x=1$. Then calculate the area using definite integral from 1 to e of the function $x^{2}\ln x$. Use integration by parts to compute the integral.
<hr style="border:1px solid red" >

+++

<br>

+++ <span style="color:green">Solutions</span>

<hr style="border:1px solid red" >

The curve is $y=x^{2}\ln x$. It meets the $x$-axis when $\ln x=0$, i.e. at $x=1$. Hence the region $R$ above the $x$-axis bounded by the curve, the $x$-axis and the line $x=\mathrm{e}$ has area

$$
\text{Area}(R)=\int_{1}^{\mathrm{e}} x^{2}\ln x\,dx.
$$


Compute the integral by parts. Let $u=\ln x$ and $dv=x^{2}\,dx$. Then $du=\dfrac{dx}{x}$ and $v=\dfrac{x^{3}}{3}$, so

$$
\int x^{2}\ln x\,dx
= \frac{x^{3}}{3}\ln x-\int \frac{x^{3}}{3}\cdot\frac{1}{x}\,dx
= \frac{x^{3}}{3}\ln x-\frac{1}{3}\int x^{2}\,dx
= \frac{x^{3}}{3}\ln x-\frac{1}{9}x^{3}+C.
$$


Therefore

$$
\text{Area}(R)
=\left[\frac{x^{3}}{3}\ln x-\frac{1}{9}x^{3}\right]_{1}^{\mathrm{e}}
=\left(\frac{\mathrm{e}^{3}}{3}\cdot 1-\frac{1}{9}\mathrm{e}^{3}\right)
-\left(0-\frac{1}{9}\right).
$$

Simplifying,

$$
\text{Area}(R)=\frac{2}{9}\mathrm{e}^{3}+\frac{1}{9}.
$$


Thus the area can be written as $p\mathrm{e}^{3}+q$ with

$$
p=\frac{2}{9},\qquad q=\frac{1}{9}.
$$

<hr style="border:1px solid red" >

+++

<br>


7. In this question you must show detailed reasoning If $\mathrm{f}(x)=\frac{x}{(x+1)(x+2)}$, evaluate $\int_{0}^{2} \mathrm{f}(x) \mathrm{d} x$, giving your answer in simplified logarithmic form.

<br>

+++ <span style="color:blue">Hint</span>

<hr style="border:1px solid red" >

 We compute the integral by partial fractions. Assume:
 $$
\frac{x}{(x+1)(x+2)}=\frac{A}{x+1}+\frac{B}{x+2}.
$$

Then work out the values of $A$ and $B$. Then integrate both parts (terms).

<hr style="border:1px solid red" >

+++

<br>

+++ <span style="color:green">Solutions</span>

<hr style="border:1px solid red" >

Assume

$$
\frac{x}{(x+1)(x+2)}=\frac{A}{x+1}+\frac{B}{x+2}.
$$

Multiplying both sides by $(x+1)(x+2)$ gives

$$
x = A(x+2)+B(x+1) = (A+B)x + (2A+B).
$$

Equating coefficients yields the linear system

$$
A+B=1,\qquad 2A+B=0.
$$

Subtracting the second equation from the first gives $-A=1$, so $A=-1$. Then $B=1-A=2$. Thus

$$
\frac{x}{(x+1)(x+2)} = -\frac{1}{x+1}+\frac{2}{x+2}.
$$

<br>

Now integrate termwise:

$$
\int_{0}^{2} f(x)\,dx
= \int_{0}^{2}\Big(-\frac{1}{x+1}+\frac{2}{x+2}\Big)\,dx
= \Big[-\ln(x+1)+2\ln(x+2)\Big]_{0}^{2},
$$

where absolute-value signs are unnecessary because $x+1>0$ and $x+2>0$ on $[0,2]$.

Evaluate the bounds:

$$
\begin{aligned}
\Big[-\ln(x+1)+2\ln(x+2)\Big]_{0}^{2}
&= \big(-\ln 3 + 2\ln 4\big) - \big(-\ln 1 + 2\ln 2\big) \\
&= -\ln 3 + 2\ln 4 - 2\ln 2 \\
&= 2(\ln 4-\ln 2)-\ln 3 \\
&= 2\ln 2 - \ln 3 \\
&= \ln 4 - \ln 3 \\
&= \ln\!\left(\frac{4}{3}\right).
\end{aligned}
$$


Therefore,

$$
\displaystyle \int_{0}^{2} \frac{x}{(x+1)(x+2)}\,dx = \ln\!\left(\frac{4}{3}\right)
$$

<hr style="border:1px solid red" >

+++

<br>

</div>
<p align="center">
<img src="/images/han.png" alt="drawing" width="500"/>
</p>
