<p align="center">
<img src="/images/iamlegend.jpg" alt="drawing" width="500"/>
</p>

<div style="font-size: 22px;">

<br><br>

1. (a) By considering turning points, show that $x^{3}-3 x^{2}+5=0$ has only one real root and that this root lies between -2 and -1

<br>

+++ <span style="color:blue">Hint</span>

<hr style="border:1px solid red" >

<hr style="border:1px solid red" >

+++

<br>

+++ <span style="color:green">Solutions</span>

<hr style="border:1px solid red" >

<hr style="border:1px solid red" >

+++

<br>

(b) Show that this root is -1.104 , correct to 3 d.p.

<br>

+++ <span style="color:blue">Hint</span>

<hr style="border:1px solid red" >

<hr style="border:1px solid red" >

+++

<br>

+++ <span style="color:green">Solutions</span>

<hr style="border:1px solid red" >

<hr style="border:1px solid red" >

+++

<br>

2. (a) Show that the equation $\mathrm{e}^{x}=x^{3}-1$ has a real root between $x=2$ and $x=3$.
(b) Use the iterative formula $x_{n+1}=\frac{\mathrm{e}^{x_{n}}+1}{x_{n}^{2}}$, starting from $x_{0}=2$, to find two further approximations to the root.
(c) Show that the root is 2.081 correct to 3 decimal places.
3. (a) Show that the gradient of $y=2 x^{3}+4 x-5$ is always positive and deduce that the equation $2 x^{3}+4 x-5=0$ has one real root only.
(b) Show that this root lies between $x=0$ and $x=1$.
(c) Show that the equation can be rearranged into the form $x=\frac{5}{2 x^{2}+4}$.
(d) Using the iterative formula $x_{n+1}=\frac{5}{2 x_{n}^{2}+4}$ and starting from $x_{0}=1$, find the next two approximations $x_{1}$ and $x_{2}$ to the root.
(e) The diagram below shows part of the graphs of $y=x$ and $y=\frac{5}{2 x^{2}+4}$, and the position of $x_{0}$.
![](https://cdn.mathpix.com/cropped/2025_10_25_1dc8f5dcfee0b81227e5g-1.jpg?height=378&width=791&top_left_y=1738&top_left_x=691)

Copy the diagram and draw on it a staircase or cobweb diagram to illustrate how the iterations converge to the root. Indicate the positions of $x_{1}$ and $x_{2}$ on the $x$-axis. [2]
(f) Show that the root is 0.893 correct to 3 decimal places.
4. A sector AOB of a circle with radius 10 cm and centre O has an angle AOB of $\theta$ radians.
The chord AB divides the sector into two regions of equal area.
![](https://cdn.mathpix.com/cropped/2025_10_25_1dc8f5dcfee0b81227e5g-2.jpg?height=386&width=489&top_left_y=217&top_left_x=1376)
(a) Show that $\theta$ satisfies the equation $2 \sin \theta-\theta=0$.
(b) Use 3 iterations of the Newton-Raphson method, starting from $\theta=2$, to find the value of $\theta$ correct to 4 decimal places.
5. The diagram shows a cross-section of a tunnel. The height is measured in metres every 0.5 metres along the cross section.
![](https://cdn.mathpix.com/cropped/2025_10_25_1dc8f5dcfee0b81227e5g-2.jpg?height=452&width=769&top_left_y=856&top_left_x=1096)
(a) Use the trapezium rule to estimate the area of the cross-section.
(b) Is it an under-estimate or over-estimate?
6. An estimate is required for the integral $\int_{0}^{1} x \sqrt{x^{3}+1} \mathrm{~d} x$.

Using 5 rectangles, find overestimates and underestimates for the value of the integral.

<br>

</div>
<p align="center">
<img src="/images/iamlegend2.jpg" alt="drawing" width="500"/>
</p>
