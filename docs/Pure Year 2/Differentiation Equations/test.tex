<p align="center">
<img src="/images/madMax.webp" alt="drawing" width="500"/>
</p>

<div style="font-size: 22px;">

<br><br>

1. Find $y$ in terms of $x$ given that $\frac{\mathrm{d} y}{\mathrm{~d} x}=x(y-1)$.

<br>

+++ <span style="color:blue">Hint</span>

<hr style="border:1px solid red" >

<hr style="border:1px solid red" >

+++

<br>

+++ <span style="color:green">Solutions</span>

<hr style="border:1px solid red" >

<hr style="border:1px solid red" >

+++

<br>

2. Solve $(x-1) \frac{\mathrm{d} y}{\mathrm{~d} x}=x y$ for $x>1$ and $y>0$, given that $y=1$ when $x=3$.
3. Obtain a particular solution to $\left(1-\mathrm{e}^{2 y}\right) \frac{\mathrm{d} y}{\mathrm{~d} x}=\mathrm{e}^{y}$ given that $y=0$ when $x=2$.
(There is no need to express $y$ in terms of $x$ ).
4. Find an expression for $y$ in terms of $x$ given that $x^{2} \frac{\mathrm{~d} y}{\mathrm{~d} x}-y^{2}=0$.
5. At time $t$ seconds the rate of increase in the concentration of flesh eating bugs in a controlled environment is proportional to the concentration $C$ of bugs present. Initially $C=100$ bugs and after 2 seconds there are five times as many.
(a) Write down a differential equation connecting $\frac{\mathrm{d} C}{\mathrm{~d} t}, C$ and $t$ and hence find an expression for $C$ in terms of $t$.
(b) How many bugs are present after 5 seconds?
\((c)\) When will the number of bugs exceed 5000 ?
(d) Find the time at which the concentration of bugs has increased by $50 \%$ of the initial concentration.
6. Water is pouring out of a small hole in the bottom of a conical container of height 25 cm . Initially the container is full.
The rate at which the height $x$ of the water remaining in the container is given by
$$
\frac{\mathrm{d} x}{\mathrm{~d} t}=-\frac{50}{\pi} x^{-\frac{3}{2}} .
$$
(a) Solve the differential equation to find $x$ in terms of $t$.
(b) How long does it take for the container to empty completely?
7. The rate at which a body loses temperature at any instant is proportional to the amount by which the temperature of the body, at that instant, exceeds the temperature of its surroundings.
A cup of tea is in a room. The ambient temperature of the room is $20^{\circ} \mathrm{C}$ and the temperature of the tea is $84^{\circ} \mathrm{C} .5$ minutes later the tea's temperature has fallen by $32^{\circ} \mathrm{C}$.
(a) How long will it take for the tea to cool to $21^{\circ} \mathrm{C}$ ?
(b) Suggest one limitation of this model.

<br>

</div>
<p align="center">
<img src="/images/madMax2.jpeg" alt="drawing" width="500"/>
</p>
