<p align="center">
<img src="/images/madMax.webp" alt="drawing" width="500"/>
</p>

<div style="font-size: 22px;">

<br><br>

1. Find $y$ in terms of $x$ given that $\frac{\mathrm{d} y}{\mathrm{~d} x}=x(y-1)$.

+++ <span style="color:blue">Hint</span>

<div style="font-size: 22px;">

Solve by "separating variables" method: arrange the equation so that all terms involving $y$ are on one side and all terms involving $x$ are on the other side, then integrate both sides.

</div>

+++

+++ <span style="color:green">Solutions</span>

<div style="font-size: 22px;">

$$
\frac{\mathrm{d}y}{\mathrm{d}x}=x(y-1)
$$

Separate variables:

$$
\frac{\mathrm{d}y}{y-1}=x\,\mathrm{d}x
$$

Integrate both sides:

$$
\int\frac{1}{y-1}\mathrm{d}y=\int x\,\mathrm{d}x
\quad\Rightarrow\quad
\ln\lvert y-1\rvert=\frac{x^2}{2}+C,
$$

Exponentiate and solve for $y$:

$$
y-1=e^{(x^2/2+C)}\quad\Rightarrow\quad
y-1=e^C \cdot e^{x^2/2}\quad\Rightarrow\quad
y=1+e^C \cdot e^{x^2/2},
$$

where $C$ is an arbitrary constant.

</div>

+++

<br>

2. Solve $(x-1) \frac{\mathrm{d} y}{\mathrm{~d} x}=x y$ for $x>1$ and $y>0$, given that $y=1$ when $x=3$.

+++ <span style="color:green">Solutions</span>

<div style="font-size: 22px;">

$$
(x-1)\frac{\mathrm{d}y}{\mathrm{d}x}=xy,\qquad x>1,\ y>0
$$

Separate variables:

$$
\frac{1}{y}\,\mathrm{d}y=\frac{x}{x-1}\,\mathrm{d}x
=\left(1+\frac{1}{x-1}\right)\mathrm{d}x
$$

Integrate both sides:

$$
\int\frac{1}{y}\,\mathrm{d}y=\int\left(1+\frac{1}{x-1}\right)\mathrm{d}x
\quad\Rightarrow\quad
\ln y = x+\ln(x-1)+C
$$

Exponentiate (with $A=e^{C}>0$):

$$
y = e^{x+\ln(x-1)+C}=e^x\cdot e^{\ln{(x-1)}}\cdot e^C= A e^{x}(x-1)
$$

Use the initial condition $y=1,\space x=3$:

$$
1 = A e^{3}\cdot 2 \quad\Rightarrow\quad A=\frac{1}{2e^{3}}.
$$

Therefore the solution is

$$
y=\frac{x-1}{2}e^{\,x-3}
$$

</div>

+++

3. Obtain a particular solution to $\left(1-\mathrm{e}^{2 y}\right) \frac{\mathrm{d} y}{\mathrm{~d} x}=\mathrm{e}^{y}$ given that $y=0$ when $x=2$.
(There is no need to express $y$ in terms of $x$ ).

+++ <span style="color:green">Solutions</span>

<div style="font-size: 22px;">

$$
(1-e^{2y})\frac{\mathrm{d}y}{\mathrm{d}x}=e^{y}
$$

Separate variables:

$$
\frac{1-e^{2y}}{e^{y}}\,\mathrm{d}y=\mathrm{d}x
\quad\Rightarrow\quad
(e^{-y}-e^{y})\,\mathrm{d}y=\mathrm{d}x
$$

Integrate both sides:

$$
\int (e^{-y}-e^{y})\,\mathrm{d}y=\int \mathrm{d}x
\quad\Rightarrow\quad
- e^{-y}-e^{y}+C=x.
$$

Use the condition $y=0,\;x=2$:

$$
2=-e^{0}-e^{0}+C=-1-1+C\quad\Rightarrow\quad C=4.
$$

Thus a particular (implicit) solution is

$$
x=4-\bigl(e^{y}+e^{-y}\bigr)
$$

or equivalently $\;e^{y}+e^{-y}=4-x$.

</div>

+++

<br>

4. Find an expression for $y$ in terms of $x$ given that $x^{2} \frac{\mathrm{~d} y}{\mathrm{~d} x}-y^{2}=0$.

+++ <span style="color:blue">Hint</span>

<div style="font-size: 22px;">

Try to find a way to separate the variables $x$ and $y$. One way is to rearrange the equation to isolate $\frac{\mathrm{d}y}{\mathrm{d}x}$, then rewrite it in differential form to separate the variables:
$$
\frac{\mathrm{d}y}{\mathrm{d}x}=\frac{y^{2}}{x^{2}}
\quad\Longrightarrow\quad y^{-2}\,\mathrm{d}y=x^{-2}\,\mathrm{d}x
$$

</div>

+++

+++ <span style="color:green">Solutions</span>

<div style="font-size: 22px;">

$$
x^{2}\frac{\mathrm{d}y}{\mathrm{d}x}-y^{2}=0,\qquad x\neq0
$$

$$
\frac{\mathrm{d}y}{\mathrm{d}x}=\frac{y^{2}}{x^{2}}
\quad\Longrightarrow\quad
y^{-2}\,\mathrm{d}y=x^{-2}\,\mathrm{d}x
$$

$$
\int y^{-2}\,\mathrm{d}y=\int x^{-2}\,\mathrm{d}x
\quad\Rightarrow\quad
-\frac{1}{y}=-\frac{1}{x}+C
$$

$$
\frac{1}{y}=\frac{1}{x}+C_1
\quad\Rightarrow\quad
y=\frac{1}{\dfrac{1}{x}+C_1}=\frac{x}{1+C_1 x},
$$

</div>

+++

<br>

5. At time $t$ seconds the rate of increase in the concentration of flesh eating bugs in a controlled environment is proportional to the concentration $C$ of bugs present. Initially $C=100$ bugs and after 2 seconds there are five times as many.
(a) Write down a differential equation connecting $\frac{\mathrm{d} C}{\mathrm{~d} t}, C$ and $t$ and hence find an expression for $C$ in terms of $t$.

+++ <span style="color:blue">Hint</span>

<div style="font-size: 22px;">

Since the rate of increase is proportional to the concentration:

$$
\qquad \frac{\mathrm{d}C}{\mathrm{d}t}=kC,
$$

where $k$ is a constant. Solve by separating variables method. Then use the initial conditions to find the constants.

</div>

+++

+++ <span style="color:green">Solution</span>

<div style="font-size: 22px;">

Since the rate of increase is proportional to the concentration:

$$
\qquad \frac{\mathrm{d}C}{\mathrm{d}t}=kC,
$$

where $k$ is a constant.

Separate variables and integrate:

$$\frac{1}{C}\,\mathrm{d}C = k\,\mathrm{d}t$$

Integral both sides:

$$
\int \frac{1}{C}\,\mathrm{d}C = \int k\,\mathrm{d}t
\quad\Longrightarrow\quad
\ln C = kt + A, \space A\text{ is constant.}
$$

so

$$
C = e^{kt+A} = e^A \cdot e^{kt}
$$

Using the initial condition $C(t=0)=100$ gives $C_0=100$, hence

$$
100=e^A\cdot e^{k \cdot 0} = e^A \quad\Rightarrow\quad e^A=100.
$$

Now we need to find $k$, Use $C(t=2)=5\times 100=500$ to find $k$:

$$
500 = 100e^{2k}\quad\Rightarrow\quad e^{2k}=5
\quad\Rightarrow\quad k=\tfrac{1}{2}\ln 5.
$$


Therefore

$$
C(t)=100e^{\frac{1}{2}\ln 5 \cdot t} = 100(e^{(\frac{1}{2}\ln 5)})^{t} = 100 e^{\ln(5^\frac{1}{2})^{t}} = 100 (5^\frac{1}{2})^t=  100\cdot 5^{t/2}
$$
$$
( \text{Note that: } e^{\ln{5^\frac{1}{2}}} = 5^\frac{1}{2} )
$$

</div>

+++

(b) How many bugs are present after 5 seconds?

+++ <span style="color:green">Solutions</span>

<div style="font-size: 22px;">

$$
C(t)=100\cdot 5^{t/2}
$$

$$
C(5)=100\cdot 5^{5/2}=100\cdot 25\sqrt{5}=2500\sqrt{5}\approx 5590.17
$$

</div>

+++

\((c)\) When will the number of bugs exceed 5000 ?

+++ <span style="color:green">Solutions</span>

<div style="font-size: 22px;">

We have $C(t)=100\cdot 5^{t/2}$. We want the time $t$ when $C(t)>5000$.

$$
100\cdot 5^{t/2}>5000
$$

$$
5^{t/2}>50
$$

Taking logarithms (or using base‑5 logs),

$$
\frac{t}{2}>\log_{5}50 \quad\Longrightarrow\quad t>2\log_{5}50.
$$

Equivalently (natural logs),

$$
t>\frac{2\ln 50}{\ln 5}\approx 4.86135\ \text{seconds}.
$$

So the number of bugs exceeds 5000 after approximately $4.86$ seconds.

</div>

+++

(d) Find the time at which the concentration of bugs has increased by $50 \%$ of the initial concentration.

+++ <span style="color:green">Solutions</span>

<div style="font-size: 22px;">

We have $C(t)=100\cdot 5^{t/2}$. We want the time $t$ such that the concentration has increased by $50\%$ of the initial value, i.e. $C(t)=150$.

$$
100\cdot 5^{t/2}=150
$$

$$
5^{t/2}=\tfrac{3}{2}
$$

$$
\frac{t}{2}=\log_{5}\!\left(\tfrac{3}{2}\right)
\quad\Longrightarrow\quad
t=2\log_{5}\!\left(\tfrac{3}{2}\right)
= \frac{2\ln(3/2)}{\ln 5}
$$

$$
t=\frac{2\ln(3/2)}{\ln 5}\approx 0.5033\ \text{seconds}
$$

</div>

+++

<br>

6. Water is pouring out of a small hole in the bottom of a conical container of height 25 cm . Initially the container is full.
The rate at which the height $x$ of the water remaining in the container is given by
$$
\frac{\mathrm{d} x}{\mathrm{~d} t}=-\frac{50}{\pi} x^{-\frac{3}{2}} .
$$
(a) Solve the differential equation to find $x$ in terms of $t$.

+++ <span style="color:green">Solutions</span>

<div style="font-size: 22px;">

$$
\frac{dx}{dt} = -\frac{50}{\pi} x^{-3/2}
$$

Separate variables:

$$
x^{3/2}\,dx = -\frac{50}{\pi}\,dt
$$

Integrate:

$$
\int x^{3/2}\,dx = \int -\frac{50}{\pi}\,dt
\quad\Longrightarrow\quad
\frac{2}{5}x^{5/2} = -\frac{50}{\pi}t + C
$$

Multiply both sides by $\tfrac{5}{2}$:

$$
x^{5/2} = -\frac{125}{\pi}t + C_1
$$

Using the initial condition $x(0)=25$ gives $C_1 = 25^{5/2}=3125$. Hence

$$
x^{5/2} = 3125 - \frac{125}{\pi}t
$$

Therefore

$$
x(t)=\left(3125-\frac{125}{\pi}t\right)^{2/5}
$$

The container empties when

$$
3125-\frac{125}{\pi}t=0 \quad\Longrightarrow\quad t=25\pi
$$

</div>

+++

<br>

(b) How long does it take for the container to empty completely?

+++ <span style="color:green">Solutions</span>

<div style="font-size: 22px;">

$$
x^{5/2}=3125-\frac{125}{\pi}t
$$


Setting $x=0$ gives

$$
0=3125-\frac{125}{\pi}t
\quad\Longrightarrow\quad
t=\frac{3125\pi}{125}=25\pi
$$

Numerically,

$$
t=25\pi\approx 78.54\ \text{seconds}.
$$

</div>

+++

<br>

7. The rate at which a body loses temperature at any instant is proportional to the amount by which the temperature of the body, at that instant, exceeds the temperature of its surroundings.
A cup of tea is in a room. The ambient temperature of the room is $20^{\circ} \mathrm{C}$ and the temperature of the tea is $84^{\circ}$. $\mathrm{C} .5$ minutes later the tea's temperature has fallen by $32^{\circ} \mathrm{C}$.
(a) How long will it take for the tea to cool to $21^{\circ} \mathrm{C}$ ?

+++ <span style="color:green">Solutions</span>

<div style="font-size: 22px;">

$$
\text{The cooling model is (Newton's law of cooling):}\qquad \frac{dT}{dt} = -k\bigl(T-20\bigr)
$$

Separate variables and integrate:
$$
\frac{dT}{T-20} = -k \cdot dt \quad\Longrightarrow\quad \ln|T-20| = -kt + C.
$$
$$
T \geq 20 \text{ so } |T-20|=T-20, \text{ thus: } \ln(T-20) = -kt + C
$$

Need to find $C$:

We know that at $t=0$, $T=84$, so:

$$
T(0)=84=20 + C e^{-k\cdot 0} = 20 + C \quad\Longrightarrow\quad C=64.
$$

(General solution: $T(t)=20 + (T_0-20)e^{-kt}$)

So with $T_0=84$, this becomes

$$
T(t)=20 + 64e^{-kt}.
$$

Use the information at $t=5$ minutes: the temperature has fallen by $32^\circ\mathrm{C}$, so $T(5)=84-32=52$. Thus

$$
52 = 20 + 64e^{-5k} \quad\Longrightarrow\quad 32 = 64e^{-5k}
$$

$$
e^{-5k}=\tfrac{1}{2}\quad\Longrightarrow\quad -5k=\ln\!\tfrac{1}{2}=-\ln 2
\quad\Longrightarrow\quad k=\frac{\ln 2}{5}.
$$

Find $t$ when $T(t)=21$:

$$
21 = 20 + 64e^{-kt}\quad\Longrightarrow\quad 1 = 64e^{-kt}
$$

$$
e^{-kt}=\tfrac{1}{64} = 2^{-6}\quad\Longrightarrow\quad -kt=-6\ln 2
$$

$$
t=\frac{6\ln 2}{k} = \frac{6\ln 2}{(\ln 2)/5}=6\cdot 5 = 30\ \text{minutes}.
$$

$$
t=30\ \text{minutes}
$$

</div>

+++

(b) Suggest one limitation of this model.

+++ <span style="color:green">Solutions</span>

<div style="font-size: 22px;">

It requires that the ambient temperature remains constant. In reality, the ambient temperature may vary over time.

</div>

+++

<br>

</div>
<p align="center">
<img src="/images/madMax2.jpeg" alt="drawing" width="500"/>
</p>
