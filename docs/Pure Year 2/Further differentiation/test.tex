<p align="center">
<img src="/images/good.png" alt="drawing" width="500"/>
</p>

<div style="font-size: 22px;">

<br><br>

1. **Differentiate the following functions**

(a) $y=\sin ^{2} x$

<br>

+++ <span style="color:blue">Hint</span>

<hr style="border:1px solid red" >

we apply chain rule

\begin{equation}
\begin{aligned}
function\ 1\ is:\ &y=\sin x \Rightarrow \frac{d y}{d x}=\cos x \\
function\ 2\ is:\ &y=x^2 \Rightarrow \frac{d\left(x^2\right)}{d x}=2 x
\end{aligned}
\end{equation}

<hr style="border:1px solid red" >

+++

<br>

+++ <span style="color:green">Solutions</span>

<hr style="border:1px solid red" >

\begin{equation}
\begin{aligned}
\frac{d\left(\sin ^2 x\right)}{d x} &= \frac{d \sin x }  {d x} \cdot 2 \cdot \sin x \\
&=2 \cdot \sin x \cdot \cos x \\
&=2 \sin (2 x)&&
\end{aligned}
\end{equation}

<hr style="border:1px solid red" >

+++

<br>

(b) $y=x^{2} \cos x$

<br>

+++ <span style="color:blue">Hint</span>

<hr style="border:1px solid red" >

we apply product rule

\begin{equation}
\begin{aligned}
function\ 1\ is:\ &y=x^2 \\
function\ 2\ is:\ &y=\cos x \\
\end{aligned}
\end{equation}

<hr style="border:1px solid red" >

+++

<br>

+++ <span style="color:green">Solutions</span>

<hr style="border:1px solid red" >

\begin{equation}
\begin{aligned}
\frac{d(x^2\cos x)}{dx} &=\cos x \frac{dx^2}{dx} + x^2\frac{d(\cos x)}{dx} \\
&=2 x \cos x + (-\sin x) x^2 \\
&=2 x\cos x - x^2 \sin x&&
\end{aligned}
\end{equation}

<hr style="border:1px solid red" >

+++

<br>

\((c)\) $y=\frac{x}{\tan 2 x}$


<br>

+++ <span style="color:blue">Hint</span>

<hr style="border:1px solid red" >

we first apply chain rule on $\tan (2x)$, then apply quotient rule:

\begin{equation}
\begin{aligned}
function\ 1\ is:\ &y=x \\
function\ 2\ is:\ &y=\tan (2x) \\
\end{aligned}
\end{equation}

<hr style="border:1px solid red" >

+++

<br>

+++ <span style="color:green">Solutions</span>

<hr style="border:1px solid red" >

\begin{equation}
\begin{aligned}
\because\ \frac{d(tan(2x))}{dx} &=2\sec^{2} (2x) \\
\therefore \frac{  d(\frac{x}{\tan 2x})  }{dx} &= \frac{\tan(2x)\frac{d(x)}{dx}-x\frac{\tan(2x)}{dx}}{\tan ^{2} 2x} \\
&= \frac{\tan(2x) - x\cdot2\cdot\sec^{2}(2x)}{\tan ^{2} 2x} \\
&= \frac{\tan(2x)}{\tan^{2}2x}-2x\cdot\frac{\sec^(2x)}{{\tan ^{2} 2x}}\\
&= \frac{1}{\tan ^{2} 2x} - 2x\cdot\frac{1}{\cos^{2}(2x)}\cdot\frac{1}{\tan ^{2} (2x)}\\
&= \cot(2x)-2x\cdot\frac{1}{\cos^{2}(2x)}\cdot\frac{\cos^{2}(2x)}{\sin^{2}(2x)}\\
&= \cot(2x)-2x\cdot\csc(2x)&&
\end{aligned}
\end{equation}

<hr style="border:1px solid red" >

+++

<br>

(d) $y=a^{2 x^{2}}$ where $a$ is a positive constant greater than 1 .

<br>

+++ <span style="color:blue">Hint</span>

<hr style="border:1px solid red" >

We can use chain rule:
function 1 is: $y=a^x$
function 2 is: $y=2 x^2$

Remember the formula for differentiation of log of any base:
\begin{equation}
\begin{aligned}
\frac{d\left(a^x\right)}{d x}=a^x \cdot \ln a
\end{aligned}
\end{equation}

<hr style="border:1px solid red" >

+++

<br>

+++ <span style="color:green">Solutions</span>

<hr style="border:1px solid red" >

\begin{equation}
\begin{aligned}
\frac{d\left(a^{2 x^2}\right)}{d x} & =\frac{d\left(2 x^2\right)}{d x} \cdot (a^{2 x^2} \cdot \ln a) \\
& =2 \cdot 2 \cdot x \cdot a^{2 x^2} \cdot \ln a \\
& =4 x \cdot \ln a \cdot a^{2 x^2} \\
& =4 a^{2 x^2} \cdot x \ln a
\end{aligned}
\end{equation}

<hr style="border:1px solid red" >

+++

<hr style="border:1px solid grey" >


2.
(a) Show from first principles that for the curve $y=\sin x$, where $x$ is in radians,

$$
\frac{\mathrm{d} y}{\mathrm{~d} x}=\cos x
$$

You may use the formula for $\sin (A \pm B)$ without proof and assume that as $h \rightarrow 0$,
$$
\frac{\cos h-1}{h} \rightarrow 0 \text { and } \frac{\sin h}{h} \rightarrow 1
$$

<br>

+++ <span style="color:blue">Hint</span>

<hr style="border:1px solid red" >

Start with expanding $\sin(A \pm B)$ using the compound angle formulae. $B$ will be $\Delta x$, the very small changes of angle in radian.

<hr style="border:1px solid red" >

+++

<br>

+++ <span style="color:green">Solutions</span>

<hr style="border:1px solid red" >


From first principle: $Q^2$.
$$
\begin{aligned}
\frac{d(\sin x)}{d x} & =\lim _{\Delta x \rightarrow 0} \frac{\sin (x+\Delta x)-\sin (x)}{\Delta x} \\
& =\lim _{\Delta x \rightarrow 0} \frac{\sin x \cdot \cos \Delta x+\cos x \cdot \sin \Delta x-\sin x}{\Delta x} \text { (expending $sin (x+\Delta x)$) } \\
& =\lim _{\Delta x \rightarrow 0} \frac{\sin x(\cos \Delta x-1)+\cos x \cdot \sin \Delta x}{\Delta x} \text { (based on Combined Angle Identity) } \\
& =\frac{\lim _{\Delta x \rightarrow 0}}{\Delta x}\left(\frac{\sin (\cos \Delta x-1)}{\Delta x}+\frac{\cos x \cdot \sin \Delta x)}{\Delta x}\right) \\
\because & \Delta x \rightarrow 0, \frac{\cos \Delta x-1}{\Delta x}=0\\
& \text { and } \Delta x \rightarrow 0, \frac{\sin \Delta x}{\Delta x}=1 \quad\\
\therefore \frac{d \sin x}{d x} &= \lim _{\Delta x \rightarrow 0}\left(\frac{\sin (\cos \Delta x-1)}{\Delta x}+\frac{\cos x \cdot \sin \Delta x}{\Delta x}\right) \\
& =\sin \cdot 0+\cos \cdot 1 \\
& =\cos x
\end{aligned}
$$

<hr style="border:1px solid red" >

+++

<br>

(b) Identify where in the proof this assumption that $x$ is in radians is needed.

<br>

+++ <span style="color:green">Solutions</span>

<hr style="border:1px solid red" >

Based on small angle approximation, $x$ need to be in radian, because. $\Delta x \rightarrow 1, \space\frac{\sin \Delta x}{\Delta x}=1$
It can only be true when $x$ is in radian, it is not true when $x$ is in angle.

<br>

<hr style="border:1px solid red" >

+++

<br>

<hr style="border:1px solid grey" >

3.
(a) Find an expression in terms of $x$ and $y$ for the gradient of the curve

<br>

$$
x^{2}+k x y+4 y^{2}=48
$$

<br>

+++ <span style="color:blue">Hint</span>

<hr style="border:1px solid red" >

Differentiation of Implicit function, together with chain rule and product rule. For example, $\frac{d(kxy)}{dx} = (ky\cdot\frac{dx}{dx} + kx\cdot \frac{d y}{d x}) = (ky + kx\cdot \frac{d y}{d x})$ 

<hr style="border:1px solid red" >

+++

<br>

+++ <span style="color:green">Solutions</span>

<hr style="border:1px solid red" >


$$
\begin{aligned}
\frac{d\left(x^2+k x y+4 y^2\right)}{d x} & =\frac{d(48)}{d x} \\
& =0 \\
& =2 x+ (ky + kx\cdot \frac{d y}{d x}) +4 \cdot 2 y \frac{d y}{d x} \\
0 & =2 x+k y+(k x+8 y) \frac{d y}{d x} \\
\text { rearrange: } \frac{d y}{d x} & =\frac{-(2 x+k y)}{k x+8 y}
\end{aligned}
$$

<br>

<hr style="border:1px solid red" >

+++

<br>

(b) The diagram shows the curve where $k=2$.

<p align="center">
<img src="https://cdn.mathpix.com/cropped/2025_10_05_85d57a8dd1c2922b6613g-1.jpg?height=410&width=566&top_left_y=1364&top_left_x=505" alt="drawing" width="300"/>
</p>

<br>

Find the values of $y$ for which the tangent to the curve is vertical.

<br>

+++ <span style="color:blue">Hint</span>

<hr style="border:1px solid red" >

$$
\begin{aligned}
\because \text{tangent is vertical, denominator of derivative function need to be 0}\\
\therefore kx+8 y & =0
\end{aligned}
$$

<hr style="border:1px solid red" >

+++

<br>

+++ <span style="color:green">Solutions</span>

<hr style="border:1px solid red" >

$$
k=2 \rightarrow \frac{d y}{d x}=\frac{-(2 x+2 y)}{2 x+8 y}=\frac{-(x+y)}{x+4 y}\\
$$

<br>

If tangent is vertical, then the denominator of the derivative (gradient) function will be 0. That means:

<br>

$$
\begin{aligned}
x+4 y & =0 \\
y & =-\frac{x}{4}
\end{aligned}
$$

<br>

Combined with original curve, we can have following equation set:

<br>

$$
\begin{aligned}
\left\{\begin{array}{l}
x^2+2 x y+4 y^2=48 \space\space\space\text {(original curve when $k=2$.)}\\
y=-\frac{x}{4}\\
\end{array}\right.\\
\end{aligned}
$$

<br>

$$
\begin{aligned}
\text{solve above:}\\
x^2+2 x \cdot\left(\frac{-x}{4}\right)+4\left(\frac{-x}{4}\right)^2&=48\\
x^2-\frac{x^2}{2}+\frac{x^2}{4}&=48\\
3 x^2&=48 \cdot 4\\
x &= \pm 8\\
y &= \pm 2
\end{aligned}
$$

<br>

<hr style="border:1px solid red" >

+++

<br>

\((c)\) Explain what happens to the gradient when $k=4$.

<br>

+++ <span style="color:blue">Hint</span>

<hr style="border:1px solid red" >

Substitute $k=4$ into original function, and rearranged, then see what kind of function it is
<hr style="border:1px solid red" >

+++

<br>

+++ <span style="color:green">Solutions</span>

<hr style="border:1px solid red" >

$$
\begin{aligned}
\text{Substitute $k=4$ into original function:}\\
x^2+4 x y+4 y^2&=48 \\
(x+2 y)^2&=48 \\
x+2 y&= \pm \sqrt{48} \\
x+2 y &= \pm 4 \sqrt{3} \\
y&=\frac{-x \pm 4 \sqrt{3}}{2} \\
y&=-\frac{1}{2} x \pm 2 \sqrt{3}
\end{aligned}
$$

<br>

$$
\text{when $k=4$, the curve of the original function contains two parallel lines with gradient of $-\frac{1}{2}$}
$$

<br>

<hr style="border:1px solid red" >

+++

<br>

<hr style="border:1px solid grey" >

4. **In this question you must show detailed reasoning**

Show that the curve $y=x-\ln x$ has one turning point only, and give the coordinates of this point.

<br>

+++ <span style="color:blue">Hint</span>

<hr style="border:1px solid red" >

$$
\begin{aligned}
\frac{d(\ln x)}{dx}=\frac{1}{x}
\end{aligned}
$$

<hr style="border:1px solid red" >

+++

<br>

+++ <span style="color:green">Solutions</span>

<hr style="border:1px solid red" >

$$
\begin{aligned}
\frac{d(x-\ln x)}{d x}=1-\frac{1}{x}\\
\end{aligned}
$$

$$
\begin{aligned}
\newline
\text{To find the turning point, set derivative function to 0:}\\
1-\frac{1}{x}=0\\
x=1 \\
\text{Because thee is only one solution, so only one turning point}
\end{aligned}
$$

$$
\begin{aligned}
x&=1\\
y&=1-\ln 1\\
y&=1\\
\text{the point is ${(x=1, \space y=1)}$}
\end{aligned}
$$

<br>

<hr style="border:1px solid red" >

+++

<br>

<hr style="border:1px solid grey" >

5. **In this question you must show detailed reasoning**

A curve has $y=\mathrm{e}^{2 x} \cos x$

(a) Show that the turning points of the curve occur at the points for which $\tan x=2$.

<br>

+++ <span style="color:blue">Hint</span>

<hr style="border:1px solid red" >

Get the differentiation (gradient) function, then set it to 0. Then rearrange and simplify.

<hr style="border:1px solid red" >

+++

<br>

+++ <span style="color:green">Solutions</span>

<hr style="border:1px solid red" >


$$
\begin{aligned}
\frac{d\left(e^{2 x} \cdot \cos x\right)}{dx} & =\frac{d\left(e^{2 x}\right)}{d x} \cdot \cos x+e^{2 x} \cdot \frac{d \cos x}{d x} \\
& =2\cdot e^{2 x} \cdot \cos x - e^{-2 x} \cdot \sin x\\
&=e^{2x}(2\cos {x}-\sin{x})
\end{aligned}
$$

<br>

$$
\begin{aligned}
\text{to work out the turn points, set derivative to 0:}
\newline
e^{2x}(2\cos {x}-\sin{x})&=0\\
\text{since $e^{2x} > 0$}\\
2\cos {x}-\sin{x}&=0\\
2&=\frac{\sin x}{\cos x}\\
2&=\tan x
\end{aligned}
$$

<br>

<hr style="border:1px solid red" >

+++

<br>

(b) Find the equation of the normal to the curve at the point for which $x=0$.

<br>

+++ <span style="color:blue">Hint</span>

<hr style="border:1px solid red" >

Find the gradient at $x=0$, then the normal will be negative reciprocal

<hr style="border:1px solid red" >

+++

<br>

+++ <span style="color:green">Solutions</span>

<hr style="border:1px solid red" >

$$
\begin{aligned}
\text{The derivative of original function is:}\\
y'&=e^{2x}(2\cos {x}-\sin{x})\\
\text{When $x=0$:}\\
y'&=e^0(2\cos 0 - \sin 0)\\
&=2
\end{aligned}
$$

$$
\begin{aligned}
\text{2 is gradient at $x=0$, then the normal will be: $-\frac{1}{2}$}\\
\text{the line equation for normal will be:}\\
y=-\frac{1}{2}x+C\\
\text{To work out C, we substitute $x=0$ into original function:}\\
y=e^{2x}\cos{x}\\
y=e^0\cdot\cos{0}=1\\
\text{substitute point ($x=0,\space y=1$), into normal line equation:}\\
1=C\\
\text{the final equation is:}\\
y=-\frac{1}{2} x +1\\
\end{aligned}
$$

<br>

<hr style="border:1px solid red" >

+++

</div>

<p align="center">
<img src="/images/goodbad.webp" alt="drawing" width="500"/>
</p>
