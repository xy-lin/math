<p align="center">
<img src="/images/leonTeaching.png" alt="drawing" width="500"/>
</p>

<div style="font-size: 22px;">

<br><br>

1. The points $\mathrm{A}, \mathrm{B}, \mathrm{C}$ and D have coordinates $(2,1,3),(4,1,5),(2,5, p)$ and $(q, r, 1)$ respectively. If $\overrightarrow{\mathrm{AB}}=\overrightarrow{\mathrm{CD}}$ what are the values of $p, q$ and $r$ ?

+++ <span style="color:green">Solutions</span>

<hr style="border:1px solid red" >

$$
A=(2,1,3),\qquad B=(4,1,5),\qquad C=(2,5,p),\qquad D=(q,r,1).
$$

Compute the vectors

$$
\overrightarrow{AB}=B-A=(4-2,\;1-1,\;5-3)=(2,0,2),
$$

and

$$
\overrightarrow{CD}=D-C=(q-2,\;r-5,\;1-p).
$$

The condition $\overrightarrow{AB}=\overrightarrow{CD}$ gives the following equations:

$$
q-2=2,\qquad r-5=0,\qquad 1-p=2.
$$

Solving these,

$$
q=4,\qquad r=5,\qquad p=-1.
$$

<hr style="border:1px solid red" >

+++

<br>

2. Points A and B have coordinates $(2,1,1)$ and $(20,-5,13)$ respectively. If point $C$ is such that $2 \overrightarrow{A C}=\overrightarrow{C B}$, what are the coordinates of $C$ ?

+++ <span style="color:green">Solutions</span>

<hr style="border:1px solid red" >

Points $A=(2,1,1)$ and $B=(20,-5,13)$. Let $C=(x,y,z)$. The vector condition is

$$
2\overrightarrow{AC}=\overrightarrow{CB}.
$$

Since $\overrightarrow{AC}=C-A$ and $\overrightarrow{CB}=B-C$, we have

$$
2(C-A)=B-C.
$$

Rearrange to solve for $C$:

$$
2C-2A=B-C \quad\Longrightarrow\quad 3C=B+2A \quad\Longrightarrow\quad C=\frac{B+2A}{3}.
$$

With $A=(2,1,1)$ and $B=(20,-5,13)$,

$$
2A=(4,2,2),\qquad B+2A=(24,-3,15),
$$

hence

$$
C=\frac{1}{3}(24,-3,15)=(8,-1,5).
$$


A quick check: $\overrightarrow{AC}=(6,-2,4)$, so $2\overrightarrow{AC}=(12,-4,8)=\overrightarrow{CB}$, as required.

Therefore,

$$
C=(8,-1,5)
$$

<hr style="border:1px solid red" >

+++

<br>

3. The point P has coordinates ( $-2,4,0$ ).
The point Q is such that $\overrightarrow{\mathrm{PQ}}=\left(\begin{array}{c}3 \\ -2 \\ 1\end{array}\right)$.
The point R has coordinates $(-1,1, r)$.
For which value of $r$ is PQR an equilateral triangle?

+++ <span style="color:blue">Hint</span>

<hr style="border:1px solid red" >

$$
Q=P+\overrightarrow{PQ}=( -2+3,\;4-2,\;0+1)=(1,2,1)
$$

<p align="center">
<img src="/images/vector.png" alt="drawing" width="500"/>
</p>

<hr style="border:1px solid red" >

+++

<br>

+++ <span style="color:green">Solutions</span>

<hr style="border:1px solid red" >

The given points are

$$
P=(-2,4,0),\qquad \overrightarrow{PQ}=\begin{pmatrix}3\\-2\\1\end{pmatrix},
\qquad R=(-1,1,r).
$$

Hence

$$
Q=P+\overrightarrow{PQ}=( -2+3,\;4-2,\;0+1)=(1,2,1).
$$


Compute the squared length of $\overrightarrow{PQ}$:

$$
\lVert\overrightarrow{PQ}\rVert^2=3^2+(-2)^2+1^2=9+4+1=14.
$$

Thus every side of the equilateral triangle must have squared length $14$.

Compute squared lengths to $R$:

$$
\overrightarrow{PR}=R-P=(1,-3,r),\qquad
\lVert\overrightarrow{PR}\rVert^2=1^2+(-3)^2+r^2=r^2+10,
$$


$$
\overrightarrow{QR}=R-Q=(-2,-1,r-1),\qquad
\lVert\overrightarrow{QR}\rVert^2=(-2)^2+(-1)^2+(r-1)^2=5+(r-1)^2.
$$

For $PQR$ to be equilateral we require

$$
\lVert\overrightarrow{PR}\rVert^2=\lVert\overrightarrow{PQ}\rVert^2
\quad\text{and}\quad
\lVert\overrightarrow{QR}\rVert^2=\lVert\overrightarrow{PQ}\rVert^2.
$$

From the first equation:

$$
r^2+10=14 \quad\Longrightarrow\quad r^2=4 \quad\Longrightarrow\quad r=\pm 2.
$$

From the second equation:

$$
5+(r-1)^2=14 \quad\Longrightarrow\quad (r-1)^2=9 \quad\Longrightarrow\quad r=4\ \text{or}\ r=-2.
$$

The only value common to both sets is $r=-2$.

Check: for $r=-2$,

$$
\lVert\overrightarrow{PR}\rVert^2 = (-2)^2+10=14,\qquad
\lVert\overrightarrow{QR}\rVert^2 = 5+(-3)^2=14,
$$

so all three sides have squared length $14$.

Therefore,

$$
r=-2
$$

<hr style="border:1px solid red" >

+++

<br>

4. Point $A$ has coordinates ( $2,3,6$ ). Point B has coordinates ( $8,6,8$ ). Find the point C so that $\overrightarrow{\mathrm{AB}}$ and $\overrightarrow{\mathrm{AC}}$ are in the same direction and $|\overrightarrow{\mathrm{AC}}|=77$.

+++ <span style="color:green">Solutions</span>

<hr style="border:1px solid red" >
Compute the direction vector:

$$
\overrightarrow{AB}=B-A=(8-2,\;6-3,\;8-6)=(6,3,2).
$$

Let $\overrightarrow{AC}=t\,\overrightarrow{AB}$ with $t>0$. Then

$$
\lvert\overrightarrow{AC}\rvert=|t|\,\lvert\overrightarrow{AB}\rvert=77.
$$

Compute the length of $\overrightarrow{AB}$:

$$
\lvert\overrightarrow{AB}\rvert=\sqrt{6^2+3^2+2^2}=\sqrt{36+9+4}=\sqrt{49}=7.
$$


Hence

$$
t=\frac{77}{7}=11.
$$


So

$$
\overrightarrow{AC}=11(6,3,2)=(66,33,22),
$$

and therefore

$$
C=A+\overrightarrow{AC}=(2,3,6)+(66,33,22)=(68,36,28).
$$


Quick check:

$$
\lvert\overrightarrow{AC}\rvert=\sqrt{66^2+33^2+22^2}=\sqrt{4356+1089+484}=\sqrt{5929}=77.
$$


Therefore,

$$
C=(68,36,28)
$$

<hr style="border:1px solid red" >

+++

<br>

5. Forces $\mathbf{F}_{1}=\lambda(3 \mathbf{i}-2 \mathbf{j}+\mathbf{k}) \mathrm{N}$ and $\mathbf{F}_{2}=\mu(\mathbf{i}+\mathbf{j}+3 \mathbf{k}) \mathrm{N}$, where $\lambda$ and $\mu$ are scalars, act on a box. Prove that it is not possible for their resultant force to act in the direction of $\mathbf{k}$.

+++ <span style="color:green">Solutions</span>

<hr style="border:1px solid red" >

Let $\mathbf{F}_1=\lambda(3\mathbf{i}-2\mathbf{j}+\mathbf{k})$ and $\mathbf{F}_2=\mu(\mathbf{i}+\mathbf{j}+3\mathbf{k})$, where $\lambda,\mu$ are scalars. Their resultant is

$$
\mathbf{R}=\mathbf{F}_1+\mathbf{F}_2
=\bigl(3\lambda+\mu\bigr)\mathbf{i}+\bigl(-2\lambda+\mu\bigr)\mathbf{j}+\bigl(\lambda+3\mu\bigr)\mathbf{k}.
$$


For $\mathbf{R}$ to act in the direction of $\mathbf{k}$ (i.e., to be parallel to $\mathbf{k}$ and nonzero) its $\mathbf{i}$- and $\mathbf{j}$-components must vanish:

$$
3\lambda+\mu=0,\qquad -2\lambda+\mu=0.
$$

Subtracting the second equation from the first gives

$$
(3\lambda+\mu)-(-2\lambda+\mu)=5\lambda=0,
$$

so $\lambda=0$. Substituting back yields $\mu=0$. Hence $\mathbf{R}=\mathbf{0}$.

This means the resultant force is zero vector, and contradicts the requirement that it acts in the direction of $\mathbf{k}$ which requires a nonzero vector.

<hr style="border:1px solid red" >

+++

<br>

6. The points $\mathrm{P}, \mathrm{Q}$ and R have position vectors $\left(\begin{array}{c}3 \\ -1 \\ 4\end{array}\right),\left(\begin{array}{c}2 \\ -3 \\ 0\end{array}\right)$ and $\left(\begin{array}{l}7 \\ 1 \\ 3\end{array}\right)$.
(a) Show that the triangle PQR is isosceles.
+++ <span style="color:green">Solutions</span>

<hr style="border:1px solid red" >

The position vectors of $P,Q,R$ are

$$
\mathbf{p}=\begin{pmatrix}3\\-1\\4\end{pmatrix},\qquad
\mathbf{q}=\begin{pmatrix}2\\-3\\0\end{pmatrix},\qquad
\mathbf{r}=\begin{pmatrix}7\\1\\3\end{pmatrix}.
$$


Compute the side vectors:

$$
\overrightarrow{PQ}=\mathbf{q}-\mathbf{p}
=\begin{pmatrix}2-3\\-3-(-1)\\0-4\end{pmatrix}
=\begin{pmatrix}-1\\-2\\-4\end{pmatrix},
$$


$$
\overrightarrow{PR}=\mathbf{r}-\mathbf{p}
=\begin{pmatrix}7-3\\1-(-1)\\3-4\end{pmatrix}
=\begin{pmatrix}4\\2\\-1\end{pmatrix}.
$$


Compute squared lengths to avoid unnecessary square roots:

$$
\lVert\overrightarrow{PQ}\rVert^2=(-1)^2+(-2)^2+(-4)^2=1+4+16=21,
$$


$$
\lVert\overrightarrow{PR}\rVert^2=4^2+2^2+(-1)^2=16+4+1=21.
$$


Since $\lVert\overrightarrow{PQ}\rVert^2=\lVert\overrightarrow{PR}\rVert^2$, we have

$$
\lVert\overrightarrow{PQ}\rVert=\lVert\overrightarrow{PR}\rVert,
$$

so two sides of triangle $PQR$ are equal. Therefore $\triangle PQR$ is isosceles (with vertex at $P$).

<hr style="border:1px solid red" >

+++

<br>

(b) M is the midpoint of QR . Find the position vector of M .

+++ <span style="color:green">Solutions</span>

<hr style="border:1px solid red" >

$$
\mathbf{q}=\begin{pmatrix}2\\-3\\0\end{pmatrix},\qquad
\mathbf{r}=\begin{pmatrix}7\\1\\3\end{pmatrix}
$$


Midpoint formula:

$$
\mathbf{m}=\frac{\mathbf{q}+\mathbf{r}}{2}
=\frac{1}{2}\begin{pmatrix}2+7\\-3+1\\0+3\end{pmatrix}
=\frac{1}{2}\begin{pmatrix}9\\-2\\3\end{pmatrix}
=\begin{pmatrix}9/2\\-1\\3/2\end{pmatrix}.
$$

<hr style="border:1px solid red" >

+++

<br>

\((c)\) Calculate the area of the triangle PQR .


+++ <span style="color:blue">Hint</span>

<hr style="border:1px solid red" >

The area of triangle PQR is given by half of magnitude of the cross product of two side vectors:
$$
\text{Area}(\triangle PQR)=\Vert \tfrac{1}{2}\overrightarrow{PQ}\times\overrightarrow{PR} \Vert
$$

So try to Calculate the cross product \(\overrightarrow{PQ}\times\overrightarrow{PR}\) first. Then find its magnitude. Finally divide by 2 to get the area.

<hr style="border:1px solid red" >

+++

<br>

+++ <span style="color:green">Solutions</span>

<hr style="border:1px solid red" >

Given

$$
\overrightarrow{PQ}=\mathbf{q}-\mathbf{p}=\begin{pmatrix}-1\\-2\\-4\end{pmatrix},\qquad
\overrightarrow{PR}=\mathbf{r}-\mathbf{p}=\begin{pmatrix}4\\2\\-1\end{pmatrix}.
$$

Their cross product is

$$
\overrightarrow{PQ}\times\overrightarrow{PR}=\begin{pmatrix}10\\-17\\6\end{pmatrix},
$$

so its magnitude is
$$
\Vert \overrightarrow{PQ}\times\overrightarrow{PR} \Vert
=\sqrt{10^2+(-17)^2+6^2}=\sqrt{425}=5\sqrt{17}.
$$

The area of triangle PQR is half this magnitude:
$$
\text{Area}(\triangle PQR)=\tfrac{1}{2} \Vert\overrightarrow{PQ}\times\overrightarrow{PR}\Vert
=\frac{5\sqrt{17}}{2}\approx 10.3078.
$$

<hr style="border:1px solid red" >

+++

<br>

7. The points $\mathrm{A}, \mathrm{B}, \mathrm{C}$ and D have position vectors $\left(\begin{array}{c}2 \\ -1 \\ 4\end{array}\right),\left(\begin{array}{l}-3 \\ -1 \\ 14\end{array}\right),\left(\begin{array}{c}-11 \\ 4 \\ 20\end{array}\right)$ and $\left(\begin{array}{c}-6 \\ 4 \\ 10\end{array}\right)$.
(a) Write down the vector $\overrightarrow{\mathrm{AB}}$.


+++ <span style="color:green">Solutions</span>

<hr style="border:1px solid red" >

$$
\overrightarrow{AB}
= \begin{pmatrix} -3\\ -1\\ 14 \end{pmatrix}
- \begin{pmatrix} 2\\ -1\\ 4 \end{pmatrix}
= \begin{pmatrix} -5\\ 0\\ 10 \end{pmatrix}.
$$

<hr style="border:1px solid red" >

+++

<br>

(b) Calculate the length BC

+++ <span style="color:green">Solutions</span>

<hr style="border:1px solid red" >

$$
\overrightarrow{BC} = \left(\begin{array}{c}2 \\ -1 \\ 4\end{array}\right) - \left(\begin{array}{l}-3 \\ -1 \\ 14\end{array}\right) = \left(\begin{array}{l}-8 \\ 5 \\ 6\end{array}\right)
$$

$$
\lVert \overrightarrow{BC}\rVert
= \sqrt{(-8)^2+5^2+6^2}
= \sqrt{125}
= 5\sqrt{5}
\approx 11.1803.
$$

<hr style="border:1px solid red" >

+++

<br>

\((c)\) Prove that ABCD is a rhombus.

+++ <span style="color:blue">Hint</span>

<hr style="border:1px solid red" >

Calculate the vectors \(\overrightarrow{CD}\) and \(\overrightarrow{DA}\) and \(\overrightarrow{AB}\) and \(\overrightarrow{BC}\) . Then show that opposite sides are parallel by showing that \(\overrightarrow{CD}=-\overrightarrow{AB}\) and \(\overrightarrow{DA}=-\overrightarrow{BC}\) . Next calculate the lengths of \(\overrightarrow{AB}\) and \(\overrightarrow{{BC}}\) sides and show they are equal.

<hr style="border:1px solid red" >

+++

<br>

+++ <span style="color:green">Solutions</span>

<hr style="border:1px solid red" >

$$
\mathbf{a}=\left(\begin{array}{c}2\\-1\\4\end{array}\right),\quad
\mathbf{b}=\left(\begin{array}{c}-3\\-1\\14\end{array}\right),\quad
\mathbf{c}=\left(\begin{array}{c}-11\\4\\20\end{array}\right),\quad
\mathbf{d}=\left(\begin{array}{c}-6\\4\\10\end{array}\right).
$$

$$
\overrightarrow{AB}=\mathbf{b}-\mathbf{a}=\left(\begin{array}{c}-5\\0\\10\end{array}\right),\qquad
\overrightarrow{BC}=\mathbf{c}-\mathbf{b}=\left(\begin{array}{c}-8\\5\\6\end{array}\right),
$$

$$
\overrightarrow{CD}=\mathbf{d}-\mathbf{c}=\left(\begin{array}{c}5\\0\\-10\end{array}\right),\qquad
\overrightarrow{DA}=\mathbf{a}-\mathbf{d}=\left(\begin{array}{c}8\\-5\\-6\end{array}\right).
$$

Note that

$$
\overrightarrow{CD}=-\overrightarrow{AB},\qquad \overrightarrow{DA}=-\overrightarrow{BC},
$$

so opposite sides are parallel.

Compute the side lengths:

$$
\lVert\overrightarrow{AB}\rVert=\sqrt{(-5)^2+0^2+10^2}=\sqrt{125}=5\sqrt{5},
$$

and

$$
\lVert\overrightarrow{BC}\rVert=\sqrt{(-8)^2+5^2+6^2}=\sqrt{125}=5\sqrt{5},
$$

similarly $\lVert\overrightarrow{CD}\rVert=\lVert\overrightarrow{DA}\rVert=5\sqrt{5}$.

All four sides are equal in length and opposite sides are parallel; hence ABCD is a rhombus.

<hr style="border:1px solid red" >

+++

<br>

</div>
<p align="center">
<img src="/images/leonMoving.png" alt="drawing" width="500"/>
</p>
