<p align="center">
<img src="/images/shawshank.jpg" alt="drawing" width="500"/>
</p>

<div style="font-size: 22px;">

<br><br>

1. When Charlie and Robin play a game, they toss a coin to see who goes first. If it is a head then Charlie goes first; if it is a tail then Robin goes first. Charlie and Robin play four games and each time the coin shows a head. Robin claims that the coin is definitely biased. Calculate a relevant probability, and discuss the validity of Robin's claim.

+++ <span style="color:blue">Hint</span>

<hr style="border:1px solid red" >

<hr style="border:1px solid red" >

+++

<br>

+++ <span style="color:green">Solutions</span>

<hr style="border:1px solid red" >

<hr style="border:1px solid red" >

+++

<br>

2. The events A and B are such that $P(A)=0.5, P(B)=0.4, P(A \cup B)=0.7$. Find
(a) $\mathrm{P}(A \cap B)$
(b) $\mathrm{P}\left(A^{\prime} \cap B^{\prime}\right)$
\((c)\) $\mathrm{P}\left(A^{\prime} \cup B\right)$
3. A delivery company charge differently for three sizes of parcel - small, medium and large - based on volume. They are considering changing the pricing structure to take into account the weight of the parcels, and do a survey to determine the weight distribution of the parcels of different sizes. The results are shown below.

\begin{tabular}{|c|c|c|c|c|}
\hline & Small weight & \begin{tabular}{c} 
Medium \\
weight
\end{tabular} & Large weight & Total \\
\hline Small volume & 63 & 12 & 2 & 77 \\
\hline Medium volume & 11 & 89 & 22 & 122 \\
\hline Large volume & 5 & 37 & 51 & 93 \\
\hline Total & 79 & 138 & 75 & 292 \\
\hline
\end{tabular}
(a) Show that the probability that the parcel is in the same category for volume as it is for weight is $\frac{203}{292}$.
(b) Find the probability that the parcel is in the same category for weight as it is for volume given that it is in the category of medium for volume.
\((c)\) Find the probability that the parcel is in the same category for weight as it is for volume given that it is in the category of small for weight.
(d) Find the probability that the parcel is in the category of large for volume given that it is in the same category for volume as it is for weight.
4. A survey was done of a group of students, and it was found that $\frac{1}{3}$ of them walked to school, and $\frac{2}{5}$ of them always ate breakfast. Of those that walked to school, $\frac{11}{30}$ always ate breakfast.
(a) Find the probability that a student chosen at random from the group walks to school and always eats breakfast.
(b) Find the probability that a student chosen at random from the group always has breakfast given that they do not walk to school.
\((c)\) Find the probability that a student chosen at random from the group does not walk to school and does not always eat breakfast.
(d) Determine with a reason whether the events of always eating breakfast and walking to school are independent.

5. Each month a couple go to one of the two Italian restaurants on the High Street, the pizzeria or the trattoria. If they visit the pizzeria one month, the probability that they visit the pizzeria the next month is 0.6 . If they visit the trattoria one month, then the probability that they visit the trattoria the next month is 0.5 . In January, they go to the trattoria.
(a) Draw a tree diagram to illustrate the possible choice of restaurants in February, March and April.
(b) Find the probability that the couple go to the same restaurant in February and March.
\((c)\) Find the probability that the couple go to the trattoria exactly once in February, March and April.
(d) Find the probability that the couple go to the pizzeria in February, given that they go to the pizzeria in April.
6. A test for a virus gives a positive result $98 \%$ of the time if a person has the virus, but it also gives a false positive result $1 \%$ of the time if the person does not have the virus. Random testing is carried out on the population.
(a) Calculate the probability that a person who tests positive for the virus has the virus when the proportion of people in the population with the virus is:
(i) 1 in 1000
(ii) 1 in 10
(b) Comment on the usefulness of the test in the two different situations.
\((c)\) If it is equally likely that someone who tests positive for the virus has the virus as it is that they do not have the virus, what is the proportion of people in the population with the virus?

<br>

</div>
<p align="center">
<img src="/images/shawshank2.jpg" alt="drawing" width="500"/>
</p>
