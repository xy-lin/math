<p align="center">
<img src="/images/flyOver.jpg" alt="drawing" width="500"/>
</p>

<div style="font-size: 22px;">

<br><br>

1. A particle is projected from a point on horizontal ground with a velocity of $25 \mathrm{~ms}^{-1}$ at an angle of $35^{\circ}$ to the ground.
(a) Calculate the maximum height reached by the particle.

<br>

+++ <span style="color:blue">Hint</span>

<hr style="border:1px solid red" >

<hr style="border:1px solid red" >

+++

<br>

+++ <span style="color:green">Solutions</span>

<hr style="border:1px solid red" >

<hr style="border:1px solid red" >

+++

<br>

(b) Calculate the distance from the point of projection to the point at which it lands on the ground.
2. A pebble is thrown horizontally with a velocity of $u \mathrm{~ms}^{-1}$ from the top of a vertical cliff 28 m above the sea below. The pebble lands 45 m from the foot of the cliff. Calculate the value of $u$.
3. Take $\boldsymbol{g}=\mathbf{1 0} \mathbf{m s}^{-\mathbf{2}}$ in this question. Air resistance should be neglected.

A ball is thrown from a point O with a velocity of $15 \mathrm{~ms}^{-1}$ at an angle of $\theta$ to the horizontal, where $\cos \theta=0.6$ and $\sin \theta=0.8$.

The diagram below shows $x$ - and $y$-axes drawn through O , the point of projection. The units of the axes are metres.
![](https://cdn.mathpix.com/cropped/2025_10_25_8a9526a52f9386306570g-1.jpg?height=275&width=338&top_left_y=1319&top_left_x=605)
(a) Show that, after time $t$ seconds, the position of the ball is given by
$$
\begin{equation*}
x=9 t, \quad y=12 t-5 t^{2} \tag{3}
\end{equation*}
$$
(b) Show that the equation of the ball is
$$
\begin{equation*}
y=\frac{4}{3} x-\frac{5}{81} x^{2} \tag{3}
\end{equation*}
$$
\((c)\) Hence calculate how far from the origin the ball lands.
4. A stone is projected over horizontal ground from a point O on the ground. The velocity of projection is $30 \mathrm{~ms}^{-1}$ at $40^{\circ}$ to the horizontal. The effects of air resistance should be neglected.
(a) State the modelling assumptions used in the standard projectile model.
(b) The stone is at a horizontal distance $x \mathrm{~m}$ from the point of projection and $y \mathrm{~m}$ above the ground $t$ seconds after projection. Write down an expressions for $x$ and $y$ in terms of $t$.

The stone passes directly over a wall which is at a horizontal distance of 34 m from O .
\((c)\) Find the time taken to reach the wall.

\section*{MEI A level Maths Projectiles Assessment}
(d) Determine the speed of the stone as it passes over the wall. Calculate also the angle between the direction of motion of the ball and the horizontal at that time, making it clear whether the ball is rising or falling.
5. A tennis player serves the ball with a velocity of $U \mathrm{~ms}^{-1}$ at an angle of $\alpha^{\circ}$ above horizontal from a point which is 2.5 m above the ground. The ball reaches a maximum height of 2.8 m above the ground.
(a) Show that $U^{2}=\frac{3 g}{5 \sin ^{2} \alpha}$
(b) The ball just passes over the net which is 90 cm high and 12 m from the server. Show that $120 \tan ^{2} \alpha-12 \tan \alpha-\frac{8}{5}=0$.
\((c)\) Hence calculate the values of $\alpha$ and $U$.
(d) To be a legal serve the ball must land less than 18.4 m from the server. Determine whether the serve is legal.
(e) Explain why a particle model may not be good enough to predict the flight of a tennis ball.

<br>

</div>
<p align="center">
<img src="/images/flyOver2.jpg" alt="drawing" width="500"/>
</p>