<p align="center">
<img src="/images/flyOver.jpg" alt="drawing" width="500"/>
</p>

<div style="font-size: 22px;">

<br><br>

1. A particle is projected from a point on horizontal ground with a velocity of $25 \mathrm{~ms}^{-1}$ at an angle of $35^{\circ}$ to the ground.
(a) Calculate the maximum height reached by the particle.

+++ <span style="color:blue">Hint</span>

<div style="font-size: 22px;">

Work out the vertical component of the initial velocity using trigonometry. Then use the kinematic equations to find the maximum height.
$$ v^2 = u^2 + 2 a s $$
s is maximum height, u is initial vertical velocity, v is final vertical velocity (0 at maximum height), and a is acceleration due to gravity (-9.8 m/s²).

</div>

+++

+++ <span style="color:green">Solutions</span>

<div style="font-size: 22px">

to calculate the maximum height, we first need to find the vertical component of the initial velocity. The vertical component $u_y$ can be calculated using the formula:
$$ u_y = u \sin \theta $$
where:
$u = 25 \mathrm{~ms}^{-1}$ (initial velocity)
$\theta = 35^{\circ}$ (angle of projection)
Calculating $u_y$:
$$ u_y = 25 \sin 35^{\circ} \approx 25 \times 0.574 = 14.35 \mathrm{~ms}^{-1} $$
Next, we can use the following kinematic equation to find the maximum height $H$:
$$ v^2 = u^2 + 2 a s $$
where:
$v = 0 \mathrm{~ms}^{-1}$ (final velocity at maximum height)
$u = 14.35 \mathrm{~ms}^{-1}$ (initial vertical velocity)
$a = -9.8 \mathrm{~ms}^{-2}$ (acceleration due to gravity, negative because it is acting downwards)
$s = H$ (maximum height)
Rearranging the equation to solve for $H$:
$$ 0 = (14.35)^2 + 2(-9.8)H $$
$$ 0 = 205.92 - 19.6H $$
$$ 19.6H = 205.92 $$
$$ H = \frac{205.92}{19.6} \approx 10.5 \mathrm{~m} $$
Therefore, the maximum height reached by the particle is approximately $10.5 \mathrm{~m}$.

</div>

+++

<br>

(b) Calculate the distance from the point of projection to the point at which it lands on the ground.

+++ <span style="color:blue">Hint</span>

<div style="font-size: 22px;">

To calculate the distance from the point of projection to the point at which the particle lands on the ground, we need to find the time of flight and then use it to calculate the horizontal distance traveled.
First, we calculate the time of flight using the vertical motion. The total time of flight $T$ can be found using the formula:
$$ T = \frac{2 u_y}{g} $$
where:
$u_y = 14.35 \mathrm{~ms}^{-1}$ (initial vertical velocity)
$g = 9.8 \mathrm{~ms}^{-2}$ (acceleration due to gravity)

</div>

+++

+++ <span style="color:green">Solutions</span>

<div style="font-size: 22px">

To calculate the distance from the point of projection to the point at which the particle lands on the ground, we need to find the time of flight and then use it to calculate the horizontal distance traveled.
First, we calculate the time of flight using the vertical motion. The total time of flight $T$ can be found using the formula:
$$ T = \frac{2 u_y}{g} $$
where:
$u_y = 14.35 \mathrm{~ms}^{-1}$ (initial vertical velocity)
$g = 9.8 \mathrm{~ms}^{-2}$ (acceleration due to gravity)
Calculating $T$:
$$ T = \frac{2 \times 14.35}{9.8} \approx \frac{28.7}{9.8} \approx 2.93 \mathrm{~s} $$
Next, we calculate the horizontal component of the initial velocity $u_x$ using the formula:
$$ u_x = u \cos \theta $$
where:
$u = 25 \mathrm{~ms}^{-1}$ (initial velocity)
$\theta = 35^{\circ}$ (angle of projection)
Calculating $u_x$:
$$ u_x = 25 \cos 35^{\circ} \approx 25 \times 0.819 = 20.48 \mathrm{~ms}^{-1} $$
Finally, we can calculate the horizontal distance $D$ traveled by the particle using the formula:
$$ D = u_x \times T $$
Calculating $D$:
$$ D = 20.48 \times 2.93 \approx 60.0 \mathrm{~m} $$
Therefore, the distance from the point of projection to the point at which the particle lands on the ground is approximately $60.0 \mathrm{~m}$.

</div>

+++

<br>

2. A pebble is thrown horizontally with a velocity of $u \mathrm{~ms}^{-1}$ from the top of a vertical cliff 28 m above the sea below. The pebble lands 45 m from the foot of the cliff. Calculate the value of $u$.

+++ <span style="color:blue">Hint</span>

<div style="font-size: 22px;">

To find the value of $u$, we first need to calculate the time it takes for the pebble to fall 28 m vertically. We can use the kinematic equation for vertical motion:
$$ s = ut + \frac{1}{2}gt^2 $$
where:
$s = 28 \mathrm{~m}$ (vertical distance)
$u = 0 \mathrm{~ms}^{-1}$ (initial vertical velocity, since the pebble is thrown horizontally)
$g = 9.8 \mathrm{~ms}^{-2}$ (acceleration due to gravity)
$t$ is the time in seconds.

</div>

+++

+++ <span style="color:green">Solutions</span>

<div style="font-size: 22px">

To find the value of $u$, we first need to calculate the time it takes for the pebble to fall 28 m vertically. We can use the kinematic equation for vertical motion:
$$ s = ut + \frac{1}{2}gt^2 $$
where:
$s = 28 \mathrm{~m}$ (vertical distance)
$u = 0 \mathrm{~ms}^{-1}$ (initial vertical velocity, since the pebble is thrown horizontally)
$g = 9.8 \mathrm{~ms}^{-2}$ (acceleration due to gravity)
$t$ is the time in seconds.
Rearranging the equation to solve for $t$:
$$ 28 = 0 + \frac{1}{2} \times 9.8 \times t^2 $$
$$ 28 = 4.9t^2 $$
$$ t^2 = \frac{28}{4.9} \approx 5.71 $$
$$ t \approx \sqrt{5.71} \approx 2.39 \mathrm{~s} $$
Next, we can use the horizontal motion to find the value of $u$. The horizontal distance traveled by the pebble is given by:
$$ D = ut $$
where:
$D = 45 \mathrm{~m}$ (horizontal distance)
$u$ is the horizontal velocity we need to find.
Rearranging the equation to solve for $u$:
$$ u = \frac{D}{t} $$
Calculating $u$:
$$ u = \frac{45}{2.39} \approx 18.83 \mathrm{~ms}^{-1} $$
Therefore, the value of $u$ is approximately $18.83 \mathrm{~ms}^{-1}$.

</div>

+++

<br>

3. Take $\boldsymbol{g}=\mathbf{1 0} \mathbf{m s}^{-\mathbf{2}}$ in this question. Air resistance should be neglected.

A ball is thrown from a point O with a velocity of $15 \mathrm{~ms}^{-1}$ at an angle of $\theta$ to the horizontal, where $\cos \theta=0.6$ and $\sin \theta=0.8$.

The diagram below shows $x$ - and $y$-axes drawn through O , the point of projection. The units of the axes are metres.
![](https://cdn.mathpix.com/cropped/2025_10_25_8a9526a52f9386306570g-1.jpg?height=275&width=338&top_left_y=1319&top_left_x=605)
(a) Show that, after time $t$ seconds, the position of the ball is given by

\begin{equation}
x=9 t, \quad y=12 t-5 t^{2}
\end{equation}

+++ <span style="color:blue">Hint</span>

<div style="font-size: 22px;">

To find the position of the ball after time  seconds, we need to resolve the initial velocity into its horizontal and vertical components.

</div>

+++

+++ <span style="color:green">Solutions</span>

<div style="font-size: 22px">

to find the position of the ball after time $t$ seconds, we need to resolve the initial velocity into its horizontal and vertical components.
The horizontal component of the velocity $u_x$ is given by:
$$ u_x = u \cos \theta $$
where:
$u = 15 \mathrm{~ms}^{-1}$ (initial velocity)
$\cos \theta = 0.6$
Calculating $u_x$:
$$ u_x = 15 \times 0.6 = 9 \mathrm{~ms}^{-1} $$
The vertical component of the velocity $u_y$ is given by:
$$ u_y = u \sin \theta $$
where:
$u = 15 \mathrm{~ms}^{-1}$ (initial velocity)
$\sin \theta = 0.8$
Calculating $u_y$:
$$ u_y = 15 \times 0.8 = 12 \mathrm{~ms}^{-1} $$
Now, we can find the position of the ball after time $t$ seconds.
The horizontal position $x$ after time $t$ is given by:
$$ x = u_x t = 9t $$
The vertical position $y$ after time $t$ is given by the kinematic equation:
$$ y = u_y t - \frac{1}{2} g t^2 $$
where:
$g = 10 \mathrm{~ms}^{-2}$ (acceleration due to gravity)
Calculating $y$:
$$ y = 12t - \frac{1}{2} \times 10 \times t^2 = 12t - 5t^2 $$
Therefore, the position of the ball after time $t$ seconds is given by:
$$ x = 9t, \quad y = 12t - 5t^2 $$

</div>

+++

<br>


(b) Show that the equation of the ball position is

\begin{equation}
y=\frac{4}{3} x-\frac{5}{81} x^{2}
\end{equation}

+++ <span style="color:blue">Hint</span>

<div style="font-size: 22px;">

Start with the equations for the position of the ball after time $t$ seconds:
$$ x = 9t $$
$$ y = 12t - 5t^2 $$

</div>

+++

+++ <span style="color:green">Solutions</span>

<div style="font-size: 22px">

to show that the equation of the ball position is given by:
$$ y = \frac{4}{3} x - \frac{5}{81} x^2 $$
we start with the equations for the position of the ball after time $t$ seconds:
$$ x = 9t $$
$$ y = 12t - 5t^2 $$
From the first equation, we can express $t$ in terms of $x$:
$$ t = \frac{x}{9} $$
Substituting this expression for $t$ into the second equation for $y$:
$$ y = 12\left(\frac{x}{9}\right) - 5\left(\frac{x}{9}\right)^2 $$
Simplifying the equation:
$$ y = \frac{12x}{9} - 5\left(\frac{x^2}{81}\right) $$
$$ y = \frac{4x}{3} - \frac{5x^2}{81} $$
Therefore, we have shown that the equation of the ball position is:
$$ y = \frac{4}{3} x - \frac{5}{81} x^2 $$

</div>

+++

<br>


\((c)\) Hence calculate how far from the origin the ball lands.

+++ <span style="color:blue">Hint</span>

<div style="font-size: 22px;">

The ball lands when $y = 0$. Set the equation of the ball position to zero and solve for $x$:

</div>

+++

+++ <span style="color:green">Solutions</span>

<div style="font-size: 22px">

to calculate how far from the origin the ball lands, we need to find the value of $x$ when the ball hits the ground, which occurs when $y = 0$.
We start with the equation of the ball position:
$$ y = \frac{4}{3} x - \frac{5}{81} x^2 $$
Setting $y = 0$:
$$ 0 = \frac{4}{3} x - \frac{5}{81} x^2 $$
Factoring out $x$:
$$ x\left(\frac{4}{3} - \frac{5}{81} x\right) = 0 $$
This gives us two solutions:
1. \( x = 0 \) (the point of projection)
2. \( \frac{4}{3} - \frac{5}{81} x = 0 \)
Solving for $x$ in the second equation:
$$ \frac{5}{81} x = \frac{4}{3} $$
$$ x = \frac{4}{3} \times \frac{81}{5} $$
$$ x = \frac{324}{15} = 21.6 \mathrm{~m} $$
Therefore, the ball lands 21.6 metres from the origin.

</div>

+++

<br>

4. A stone is projected over horizontal ground from a point O on the ground. The velocity of projection is $30 \mathrm{~ms}^{-1}$ at $40^{\circ}$ to the horizontal. The effects of air resistance should be neglected.
(a) State the modelling assumptions used in the standard projectile model.

+++ <span style="color:green">Solutions</span>

<div style="font-size: 22px">

1. The only force acting on the projectile is gravity (no air resistance).
2. The acceleration due to gravity is constant and acts vertically downwards.
3. The projectile is treated as a point mass (its size and shape are neglected).
4. The ground is level and horizontal.

</div>

+++

<br>


(b) The stone is at a horizontal distance $x \mathrm{~m}$ from the point of projection and $y \mathrm{~m}$ above the ground $t$ seconds after projection. Write down an expressions for $x$ and $y$ in terms of $t$.

+++ <span style="color:blue">Hint</span>

<div style="font-size: 22px;">

To find the expressions for $x$ and $y$ in terms of $t$, we first resolve the initial velocity into its horizontal and vertical components.

</div>

+++

+++ <span style="color:green">Solutions</span>

<div style="font-size: 22px">

To find the expressions for $x$ and $y$ in terms of $t$, we first resolve the initial velocity into its horizontal and vertical components.
The horizontal component of the velocity $u_x$ is given by:
$$ u_x = u \cos \theta $$
where:
$u = 30 \mathrm{~ms}^{-1}$ (initial velocity)
$\theta = 40^{\circ}$
Calculating $u_x$:
$$ u_x = 30 \cos 40^{\circ} \approx 30 \times 0.766 = 22.98 \mathrm{~ms}^{-1} $$
The vertical component of the velocity $u_y$ is given by:
$$ u_y = u \sin \theta $$
where:
$u = 30 \mathrm{~ms}^{-1}$ (initial velocity)
$\theta = 40^{\circ}$
Calculating $u_y$:
$$ u_y = 30 \sin 40^{\circ} \approx 30 \times 0.643 = 19.29 \mathrm{~ms}^{-1} $$
Now, we can find the expressions for $x$ and $y$ in terms of $t$.
The horizontal position $x$ after time $t$ is given by:
$$ x = u_x t = 22.98t $$
The vertical position $y$ after time $t$ is given by the kinematic equation:
$$ y = u_y t - \frac{1}{2} g t^2 $$
where:
$g = 9.8 \mathrm{~ms}^{-2}$ (acceleration due to gravity)
Calculating $y$:
$$ y = 19.29t - \frac{1}{2} \times 9.8 \times t^2 = 19.29t - 4.9t^2 $$
Therefore, the expressions for $x$ and $y$ in terms of $t$ are:
$$ x = 22.98t, \quad y = 19.29t - 4.9t^2 $$


</div>

+++

<br>

The stone passes directly over a wall which is at a horizontal distance of 34 m from O .
\((c)\) Find the time taken to reach the wall.

+++ <span style="color:blue">Hint</span>

<div style="font-size: 22px;">

Use the expression for the horizontal position $x$ in terms of time $t$:
$$ x = 22.98t $$

</div>

+++

+++ <span style="color:green">Solutions</span>

<div style="font-size: 22px">

to find the time taken to reach the wall, we can use the expression for the horizontal position $x$ in terms of time $t$:
$$ x = 22.98t $$
We know that the horizontal distance to the wall is 34 m, so we can set $x = 34$ and solve for $t$:
$$ 34 = 22.98t $$
$$ t = \frac{34}{22.98} \approx 1.48 \mathrm{~s} $$
Therefore, the time taken to reach the wall is approximately $1.48$ seconds.

</div>

+++

<br>

(d) Determine the speed of the stone as it passes over the wall. Calculate also the angle between the direction of motion of the ball and the horizontal at that time, making it clear whether the ball is rising or falling.

+++ <span style="color:blue">Hint</span>

<div style="font-size: 22px;">
To determine the speed of the stone as it passes over the wall, we need to find both the horizontal and vertical components of the velocity at that time.

The angle $\phi$ between the direction of motion of the stone and the horizontal can be found using the tangent function:
$$ \tan \phi = \frac{u_y}{u_x} $$

To determine whether the ball is rising or falling, we need to look at the sign of vertical component of the velocity $u_y$ at that time.

</div>

+++

+++ <span style="color:green">Solutions</span>

<div style="font-size: 22px">

To determine the speed of the stone as it passes over the wall, we need to find both the horizontal and vertical components of the velocity at that time.
The horizontal component of the velocity $u_x$ remains constant throughout the motion:
$$ u_x = 22.98 \mathrm{~ms}^{-1} $$
The vertical component of the velocity $u_y$ at time $t$ can be found using the equation:
$$ u_y = u_{y0} - gt $$
where:
$u_{y0} = 19.29 \mathrm{~ms}^{-1}$ (initial vertical velocity)
$g = 9.8 \mathrm{~ms}^{-2}$ (acceleration due to gravity)
$t = 1.48 \mathrm{~s}$ (time to reach the wall)
Calculating $u_y$:
$$ u_y = 19.29 - 9.8 \times 1.48 $$
$$ u_y \approx 19.29 - 14.50 \approx 4.79 \mathrm{~ms}^{-1} $$
Now, we can find the speed of the stone as it passes over the wall using the Pythagorean theorem:
$$ v = \sqrt{u_x^2 + u_y^2} $$
Calculating $v$:
$$ v = \sqrt{(22.98)^2 + (4.79)^2} $$
$$ v \approx \sqrt{528.48 + 22.94} \approx \sqrt{551.42} \approx 23.48 \mathrm{~ms}^{-1} $$
To find the angle $\phi$ between the direction of motion of the stone and the horizontal, we can use the tangent function:
$$ \tan \phi = \frac{u_y}{u_x} $$
Calculating $\phi$:
$$ \tan \phi = \frac{4.79}{22.98} \approx 0.208 $$
$$ \phi \approx \tan^{-1}(0.208) \approx 11.8^{\circ} $$
Since the vertical component of the velocity $u_y$ is positive, the stone is still rising as it passes over the wall.
Therefore, the speed of the stone as it passes over the wall is approximately $23.48 \mathrm{~ms}^{-1}$, and the angle between the direction of motion of the stone and the horizontal is approximately $11.8^{\circ}$, with the stone rising.

</div>

+++

<br>

5. A tennis player serves the ball with a velocity of $U \mathrm{~ms}^{-1}$ at an angle of $\alpha^{\circ}$ above horizontal from a point which is 2.5 m above the ground. The ball reaches a maximum height of 2.8 m above the ground.
(a) Show that $U^{2}=\frac{3 g}{5 \sin ^{2} \alpha}$

+++ <span style="color:blue">Hint</span>

<div style="font-size: 22px;">

By analyzing the vertical motion of the tennis ball, we can use the kinematic equations to relate the maximum height reached by the ball to its initial vertical velocity component.
We use this equation:
$$ v^2 = u^2 + 2 a s $$
where: 
a = -g (acceleration due to gravity),
u = U sin α (initial vertical velocity),
s = H (maximum height above the point of projection),
v = 0 (final vertical velocity at maximum height).

And remember that the height above the point of projection is 2.8 m - 2.5 m = 0.3 m. So \(H = 0.3 m\).

So:

$$ 0 = (U \sin \alpha)^2 - 2gH $$

Rearranging:
$$ H = \frac{(U \sin \alpha)^2}{2g} $$

</div>

+++

+++ <span style="color:green">Solutions</span>

<div style="font-size: 22px">

to show that \( U^{2} = \frac{3g}{5 \sin^{2} \alpha} \), we start by analyzing the vertical motion of the tennis ball.
The maximum height \( H \) reached by the ball above the point of projection can be calculated using the kinematic equation:
$$ H = \frac{u_y^2}{2g} $$
where:
\( u_y = U \sin \alpha \) is the vertical component of the initial velocity,
\( g \) is the acceleration due to gravity (approximately \( 9.8 \mathrm{~ms}^{-2} \)).
Given that the ball reaches a maximum height of 2.8 m above the ground and is projected from a height of 2.5 m, the height above the point of projection is:
$$ H = 2.8 \mathrm{~m} - 2.5 \mathrm{~m} = 0.3 \mathrm{~m} $$
Substituting \( H \) and \( u_y \) into the kinematic equation:
$$ 0.3 = \frac{(U \sin \alpha)^2}{2g} $$
Rearranging the equation to solve for \( U^2 \):
$$ (U \sin \alpha)^2 = 0.3 \times 2g $$
$$ U^2 \sin^2 \alpha = 0.6g $$
Finally, dividing both sides by \( \sin^2 \alpha \):
$$ U^2 = \frac{0.6g}{\sin^2 \alpha} = \frac{3g}{5 \sin^2 \alpha} $$
Thus, we have shown that \( U^{2} = \frac{3g}{5 \sin^{2} \alpha} \).

</div>

+++

<br>

(b) The ball just passes over the net which is 90 cm high and 12 m from the server. Show that $120 \tan ^{2} \alpha-12 \tan \alpha-\frac{8}{5}=0$.

+++ <span style="color:blue">Hint</span>

<div style="font-size: 22px;">

The horizontal distance $x$ to the net is 12 m, and the height of the net $y$ is $0.9 m$ above the ground. 

The key is that the position of the ball is lower than the height of the starting point, so for the position equation the vertical displacement is negative.
$$ y = U \times \sin \alpha \times t - \frac {1}{2} \times g \times t^2 = 0.9 - 2.5 = -1.6 \mathrm{~m} $$

</div>

+++

+++ <span style="color:green">Solutions</span>

<div style="font-size: 22px">

to show that \( 120 \tan^{2} \alpha - 12 \tan \alpha - \frac{8}{5} = 0 \), we start by analyzing the motion of the tennis ball as it passes over the net.
The horizontal distance \( x \) to the net is 12 m, and the height of the net is 0.9 m above the ground. The ball is projected from a height of 2.5 m, so the height of the ball above the point of projection when it passes over the net is:
$$ y = 0.9 \mathrm{~m} - 2.5 \mathrm{~m} = -1.6 \mathrm{~m} $$
Using the equations of motion for projectile motion, we can express the horizontal and vertical positions of the ball in terms of time \( t \):
$$ x = U \cos \alpha \cdot t $$
$$ y = U \sin \alpha \cdot t - \frac{1}{2} g t^2 $$
Substituting \( x = 12 \) m into the first equation to find \( t \):
$$ t = \frac{12}{U \cos \alpha} $$
Substituting this expression for \( t \) into the second equation for \( y \):
$$ -1.6 = U \sin \alpha \cdot \frac{12}{U \cos \alpha} - \frac{1}{2} g \left( \frac{12}{U \cos \alpha} \right)^2 $$
Simplifying the equation:
$$ -1.6 = 12 \tan \alpha - \frac{1}{2} g \cdot \frac{144}{U^2 \cos^2 \alpha} $$
Substituting \( U^2 = \frac{3g}{5 \sin^2 \alpha} \) into the equation:
$$ -1.6 = 12 \tan \alpha - \frac{1}{2} g \cdot \frac{144}{\frac{3g}{5 \sin^2 \alpha} \cos^2 \alpha} $$
Simplifying further:
$$ -1.6 = 12 \tan \alpha - \frac{1}{2} g \cdot \frac{144 \cdot 5 \sin^2 \alpha}{3g \cos^2 \alpha} $$
$$ -1.6 = 12 \tan \alpha - \frac{120 \sin^2 \alpha}{\cos^2 \alpha} $$
Rearranging the equation:
$$ 120 \tan^2 \alpha - 12 \tan \alpha - \frac{8}{5} = 0 $$
Thus, we have shown that \( 120 \tan^{2} \alpha - 12 \tan \alpha - \frac{8}{5} = 0 \).

</div>

+++

<br>

\((c)\) Hence calculate the values of $\alpha$ and $U$.

+++ <span style="color:blue">Hint</span>

<div style="font-size: 22px;">

solve this quadratic equation for \( \tan \alpha \) using the quadratic formula
$$ \tan \alpha = \frac{-b \pm \sqrt{b^2 - 4ac}}{2a} $$

</div>

+++

+++ <span style="color:green">Solutions</span>

<div style="font-size: 22px">

To calculate the values of \( \alpha \) and \( U \), we start with the quadratic equation derived in part (b):
$$ 120 \tan^{2} \alpha - 12 \tan \alpha - \frac{8}{5} = 0 $$
We can solve this quadratic equation for \( \tan \alpha \) using the quadratic formula:
$$ \tan \alpha = \frac{-b \pm \sqrt{b^2 - 4ac}}{2a} $$
where:
\( a = 120 \),
\( b = -12 \),
\( c = -\frac{8}{5} \).
Calculating the discriminant:
$$ b^2 - 4ac = (-12)^2 - 4 \cdot 120 \cdot \left(-\frac{8}{5}\right) = 144 + 768 = 912 $$
Calculating \( \tan \alpha \):
$$ \tan \alpha = \frac{12 \pm \sqrt{912}}{240} $$
Calculating \( \sqrt{912} \approx 30.2 \):
$$ \tan \alpha = \frac{12 \pm 30.2}{240} $$
This gives us two possible solutions for \( \tan \alpha \):
1. \( \tan \alpha = \frac{42.2}{240} \approx 0.176 \)
2. \( \tan \alpha = \frac{-18.2}{240} \approx -0.076 \) (not physically meaningful for this context)
Taking the positive solution:
$$ \alpha = \tan^{-1}(0.176) \approx 10.0^{\circ} $$
Now, we can calculate \( U \) using the equation derived in part (a):
$$ U^{2} = \frac{3g}{5 \sin^{2} \alpha} $$
Calculating \( \sin \alpha \):
$$ \sin 10.0^{\circ} \approx 0.174 $$
Calculating \( U^{2} \):
$$ U^{2} = \frac{3 \cdot 9.8}{5 \cdot (0.174)^{2}} \approx \frac{29.4}{0.151} \approx 194.7 $$
Calculating \( U \):
$$ U \approx \sqrt{194.7} \approx 13.95 \mathrm{~ms}^{-1} $$
Therefore, the values of \( \alpha \) and \( U \) are approximately \( 10.0^{\circ} \) and \( 13.95 \mathrm{~ms}^{-1} \), respectively.

</div>

+++

<br>

(d) To be a legal serve the ball must land less than 18.4 m from the server. Determine whether the serve is legal.

+++ <span style="color:blue">Hint</span>

<div style="font-size: 22px;">

We first need to find the time of flight \( t \) using the vertical motion equation:
$$ y = U \sin \alpha \cdot t - \frac{1}{2} g t^2 $$

When ball lands, \( y = -2.5 \) m (the ball lands 2.5 m below the point of projection).

Then, we can calculate the horizontal distance \( x \) using the horizontal motion equation:
$$ x = U \cos \alpha \cdot t $$

</div>

+++

+++ <span style="color:green">Solutions</span>

<div style="font-size: 22px">

To determine whether the serve is legal, we need to calculate the horizontal distance the ball travels before landing.
Using the horizontal motion equation:
$$ x = U \cos \alpha \cdot t $$
We first need to find the time of flight \( t \) using the vertical motion equation:
$$ y = U \sin \alpha \cdot t - \frac{1}{2} g t^2 $$
Setting \( y = -2.5 \) m (the ball lands 2.5 m below the point of projection):
$$ -2.5 = 13.95 \sin 10.0^{\circ} \cdot t - \frac{1}{2} \cdot 9.8 \cdot t^2 $$
Calculating \( 13.95 \sin 10.0^{\circ} \approx 2.42 \):
$$ -2.5 = 2.42t - 4.9t^2 $$
Rearranging the equation:
$$ 4.9t^2 - 2.42t - 2.5 = 0 $$
Using the quadratic formula to solve for \( t \):
$$ t = \frac{2.42 \pm \sqrt{(-2.42)^2 - 4 \cdot 4.9 \cdot (-2.5)}}{2 \cdot 4.9} $$
Calculating the discriminant:
$$ (-2.42)^2 - 4 \cdot 4.9 \cdot (-2.5) = 5.8564 + 49 = 54.8564 $$
Calculating \( t \):
$$ t = \frac{2.42 \pm \sqrt{54.8564}}{9.8} $$
Calculating \( \sqrt{54.8564} \approx 7.41 \):
$$ t = \frac{2.42 \pm 7.41}{9.8} $$
This gives us two possible solutions for \( t \):
1. \( t = \frac{9.83}{9.8} \approx 1.00 \mathrm{~s} \)
2. \( t = \frac{-4.99}{9.8} \) (not physically meaningful for this context)
Taking the positive solution:
$$ t \approx 1.00 \mathrm{~s} $$
Now, we can calculate the horizontal distance \( x \):
$$ x = 13.95 \cos 10.0^{\circ} \cdot 1.00 $$
Calculating \( 13.95 \cos 10.0^{\circ} \approx 13.75 \):
$$ x \approx 13.75 \mathrm{~m} $$
Since the ball lands approximately 13.75 m from the server, which is less than the legal limit of 18.4 m, the serve is legal.

</div>

+++

<br>

(e) Explain why a particle model may not be good enough to predict the flight of a tennis ball.

+++ <span style="color:green">Solutions</span>

<div style="font-size: 22px">

A particle model may not be good enough to predict the flight of a tennis ball because it neglects factors such as air resistance, spin, and the ball's size and shape. Air resistance can significantly affect the trajectory of the ball, especially at high speeds. Additionally, the ball's size and shape can affect how it interacts with the air, leading to deviations from the idealized projectile motion predicted by a simple particle model.

</div>

+++

<br>

</div>
<p align="center">
<img src="/images/flyOver2.jpg" alt="drawing" width="500"/>
</p>