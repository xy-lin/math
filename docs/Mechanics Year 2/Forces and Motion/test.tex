<p align="center">
<img src="/images/johnWick.jpg" alt="drawing" width="500"/>
</p>

<div style="font-size: 22px;">

<br><br>

1. A box of mass 4 kg rests on a rough plane inclined at $20^{\circ}$ to the horizontal.
(a) Calculate the magnitude of the normal reaction between the box and the plane.

+++ <span style="color:blue">Hint</span>

<div style="font-size: 22px;">

<p align="center">
<img src="/assets/Mech2_Force_Q1.png" alt="drawing" width="500"/>
</p>

</div>

+++

+++ <span style="color:green">Solutions</span>

<div style="font-size: 22px;">

The weight of the box is given by \(W = mg = 4 \times 9.8 = 39.2 \, \text{N}\).
The normal reaction \((R)\) can be found using the component of the weight perpendicular to the inclined plane:
\[R = W \cos(20^\circ) = -39.2 \times \cos(20^\circ) \approx -36.8 \, \text{N}.\]
It is asked to calculate the magnitude of the normal reaction, so the answer is: $36.8 \, \text{N}$

</div>

+++

<br>

(b) Calculate the magnitude of the frictional force between the box and the plane, indicating the direction clearly.

+++ <span style="color:blue">Hint</span>

<div style="font-size: 22px;">

The component of the weight parallel to the inclined plane can be calculated using:
\[F_{\text{parallel}} = W \sin(20^\circ)\]

</div>

+++

+++ <span style="color:green">Solutions</span>

<div style="font-size: 22px">

The component of the weight parallel to the inclined plane is given by:
\[F_{\text{parallel}} = W \sin(20^\circ) = 39.2 \times \sin(20^\circ) \approx 13.4 \, \text{N}.\]
Since the box is at rest, the frictional force must balance this component. Therefore, the magnitude of the frictional force is \(13.4 \, \text{N}\), acting up the plane to oppose the component of weight down the plane.

</div>

+++

<br>


2. A particle P of mass 3 kg is suspended by two strings inclined at $60^{\circ}$ and $30^{\circ}$ to the horizontal as shown in the diagram. Calculate the tensions in the strings PA and PB.
![](https://cdn.mathpix.com/cropped/2025_10_25_b19637d3b44bcb94728bg-1.jpg?height=264&width=555&top_left_y=702&top_left_x=513)

+++ <span style="color:blue">Hint</span>

<div style="font-size: 22px;">

Resolve the forces vertically and horizontally for tension A and B. The vertical components must sum to the weight of the particle, and the horizontal components must balance each other.

</div>

+++

+++ <span style="color:green">Solutions</span>

<div style="font-size: 22px">

Let the tension in string PA be \(T_A\) and in string PB be \(T_B\). The weight of the particle P is given by:
\[W = mg = 3 \times 9.8 = 29.4 \, \text{N}.\]
Resolving the forces vertically and horizontally, we have:
1. Vertical components:
\[T_A \sin(60^\circ) + T_B \sin(30^\circ) = W\]
2. Horizontal components:
\[T_A \cos(60^\circ) = T_B \cos(30^\circ)\]
From the horizontal equation, we can express \(T_B\) in terms of \(T_A\):
\[T_B = T_A \frac{\cos(60^\circ)}{\cos(30^\circ)} = T_A \frac{0.5}{\sqrt{3}/2} = \frac{T_A}{\sqrt{3}}.\]
Substituting this into the vertical equation:
\[T_A \sin(60^\circ) + \left(\frac{T_A}{\sqrt{3}}\right) \sin(30^\circ) = 29.4\]
With the values of \(\sin(60^\circ) = \sqrt{3}/2\) and \(\sin(30^\circ) = 1/2\):
\[T_A \left(\frac{\sqrt{3}}{2} + \frac{1}{2\sqrt{3}}\right) = 29.4\]
Solving for \(T_A\):
\[T_A \left(\frac{2}{\sqrt{3}}\right) = 29.4\]
\[T_A = 29.4 \times \frac{\sqrt{3}}{2} \approx 25.4 \, \text{N}.\]
Now substituting \(T_A\) back to find \(T_B\):
\[T_B = \frac{25.4}{\sqrt{3}} \approx 14.7 \, \text{N}.\]
Therefore, the tensions in the strings are approximately:
- Tension in string PA: \(25.4 \, \text{N}\)
- Tension in string PB: \(14.7 \, \text{N}\)

</div>

+++

<br>

3. A sledge of mass 5 kg is pulled over a frozen lake by a string at $55^{\circ}$ to the horizontal. The tension in the string is 20 N . There is a resistance to motion of 7 N . Calculate the acceleration of the sledge.

+++ <span style="color:blue">Hint</span>

<div style="font-size: 22px;">

</div>

+++

+++ <span style="color:green">Solutions</span>

<div style="font-size: 22px">

The horizontal component of the tension in the string is given by:
\[T_{\text{horizontal}} = T \cos(55^\circ) = 20 \times \cos(55^\circ) \approx 11.47 \, \text{N}.\]
The net force acting on the sledge is:
\[F_{\text{net}} = T_{\text{horizontal}} - \text{resistance} = 11.47 - 7 = 4.47 \, \text{N}.\]
Using Newton's second law, the acceleration \(a\) of the sledge can be calculated using:
\[a = \frac{F_{\text{net}}}{m} = \frac{4.47}{5} \approx 0.894 \, \text{m/s}^2.\]
Therefore, the acceleration of the sledge is approximately \(0.894 \, \text{m/s}^2\).

</div>

+++

<br>


4. A lightbulb of mass 0.1 kg hangs from a wire. A horizontal force of $P \mathrm{~N}$ is applied to the lightbulb so that the angle between the wire and the vertical is $\alpha^{\circ}$.
(a) Draw a diagram showing all the forces on the lightbulb.

+++ <span style="color:green">Solutions</span>

<div style="font-size: 22px">

<p align="center">
<img src="/assets/Mech2_Force_Q4_a.png" alt="drawing" width="500"/>
</p>

</div>

+++

<br>



(b) In the case where $\alpha^{\circ}=30^{\circ}$, find the tension in the wire.

+++ <span style="color:green">Solutions</span>

<div style="font-size: 22px">

The weight of the lightbulb is given by:
\[W = mg = 0.1 \times 9.8 = 0.98 \, \text{N}.\]
Resolving the forces vertically and horizontally, we have:
1. Vertical components:
\[T \cos(30^\circ) = W\]
2. Horizontal components:
\[T \sin(30^\circ) = P\]
From the vertical equation, we can solve for the tension \(T\):
\[T = \frac{W}{\cos(30^\circ)} = \frac{0.98}{\sqrt{3}/2} \approx 1.13 \, \text{N}.\]


</div>

+++

<br>

\((c)\) Find the magnitude of $P$ in this situation.


+++ <span style="color:green">Solutions</span>

<div style="font-size: 22px">

Using the horizontal components equation:
\[P = T \sin(30^\circ) = 1.13 \times 0.5 \approx 0.565 \, \text{N}.\]

</div>

+++

<br>


(d) The force $P$ can be increased to a maximum value of 2 N , Find the maximum possible value for $\alpha$.

+++ <span style="color:blue">Hint</span>

<div style="font-size: 22px;">

From the horizontal and vertical components equations, we can express \alpha in terms of \(P\) and \(W\):
\[P = T \sin(\alpha)\]
\[W = T \cos(\alpha)\]

Try to eliminate \(T\) and rearrange to find \(\alpha\).

</div>

+++

+++ <span style="color:green">Solutions</span>

<div style="font-size: 22px">

Using the horizontal components equation:
\[P = T \sin(\alpha)\]
From the vertical components equation:
\[T = \frac{W}{\cos(\alpha)}\]
Substituting \(T\) into the equation for \(P\):
\[P = \frac{W}{\cos(\alpha)} \sin(\alpha) = W \tan(\alpha)\]
Rearranging for \(\alpha\):
\[\tan(\alpha) = \frac{P}{W} = \frac{2}{0.98} \approx 2.04\]
\[\alpha = \tan^{-1}(2.04) \approx 63.4^{\circ}.\]

</div>

+++

<br>



5. Two boxes of mass 20 kg and 25 kg are connected by a rope and are to be pulled up a smooth slope inclined at $15^{\circ}$ to the horizontal with the 20 kg box following the other. The 25 kg box is attached to a cable which is parallel to the plane and passes over a smooth pulley. The other end of a cable is attached to a sack of mass $M \mathrm{~kg}$ which hangs freely. The sack moves downwards with an acceleration of $0.14 \mathrm{~m} \mathrm{~s}^{-2}$.
(a) Draw a diagram to show all the forces acting on the boxes and the sack.

+++ <span style="color:green">Solutions</span>

<div style="font-size: 22px">

<p align="center">
<img src="/assets/Mech2_forces_Q5_a.jpg" alt="drawing" width="800"/>
</p>

</div>

+++

<br>


(b) Calculate the tension in the rope between the two boxes.

+++ <span style="color:blue">Hint</span>

<div style="font-size: 22px;">

Start with the equations of net force for box1, box2 and the sack. Use Newton's second law \(F = ma\) for each object and solve the equations.
Net force is given by:
- For box 1: \(T_1 - \text{weight component down slope} = m_1 a\)
- For box 2: \(T_2 - T_1 - \text{weight component down slope} = m_2 a\)
- For the sack: \(\text{weight} - T_2 = Ma\)

</div>

+++

+++ <span style="color:green">Solutions</span>

<div style="font-size: 22px">

To find the tension in the rope between the two boxes, we first need to analyze the forces acting on each box and the sack.
For the 20 kg box:
The weight component down the slope is \(20g \sin(15^\circ)\).
The net force acting on the 20 kg box is given by:
\begin{equation}
T_1 - 20g \sin(15^\circ) = 20a\label{eq:1}
\end{equation}
For the 25 kg box:
The weight component down the slope is \(25g \sin(15^\circ)\).
The net force acting on the 25 kg box is given by:
\begin{equation}
T_2 - T_1 - 25g \sin(15^\circ) = 25a
\end{equation}
For the sack:
The weight of the sack is \(Mg\).
The net force acting on the sack is given by:
\begin{equation}
Mg - T_2 = Ma
\end{equation}
We can solve these equations simultaneously to find the tensions \(T_1\) and \(T_2\). Given that the acceleration \(a = 0.14 \, \text{m/s}^2\) and \(g = 9.8 \, \text{m/s}^2\), we can substitute these values into the equations.
For the 20 kg box, we can just use equation \eqref{eq:1} to find \(T_1\):
\[T_1 - 20 \times 9.8 \times \sin(15^\circ) = 20 \times 0.14\]
\[T_1 - 50.34 = 2.8\]
\[T_1 = 53.14 \, \text{N}\]

</div>

+++

<br>




\((c)\) Calculate the value of $M$.

+++ <span style="color:blue">Hint</span>

<div style="font-size: 22px;">

Use the equation for the sack and the tension \(T_2\) found from the 25 kg box equation to solve for \(M\).
Calculate \(T_2\) first using the 25 kg box equation, then substitute into the sack equation to find \(M\).

</div>

+++

+++ <span style="color:green">Solutions</span>

<div style="font-size: 22px">

Using the tension \(T_2\) from the 25 kg box equation:
\[T_2 - T_1 - 25g \sin(15^\circ) = 25a\]
Substituting \(T_1 = 53.14 \, \text{N}\):
\[T_2 - 53.14 - 63.17 = 3.5\]
\[T_2 = 119.81 \, \text{N}\]
Now, using the sack equation:
\[Mg - T_2 = Ma\]
Substituting \(T_2 = 119.81 \, \text{N}\):
\[Mg - 119.81 = M \times 0.14\]
\[9.8M - 119.81 = 0.14M\]
Rearranging gives:
\[9.66M = 119.81\]
\[M \approx 12.4 \, \text{kg}.\]

</div>

+++

<br>

(d) In practice the acceleration of the system with these masses is less than $0.14 \mathrm{~m} \mathrm{~s}^{-2}$. Explain how the modelling assumptions need to be changed to explain this.

+++ <span style="color:green">Solutions</span>

<div style="font-size: 22px">

In practice, the acceleration of the system may be less than \(0.14 \, \text{m/s}^2\) due to several factors that were not considered in the idealized model. These factors include:
1. Friction: The slope and pulley may not be perfectly smooth, leading to frictional forces that oppose the motion of the boxes and the sack.
2. Air Resistance: The movement of the boxes and sack through the air may encounter air resistance, which would reduce the net force acting on the system.
3. Non-ideal Pulley: The pulley may have some  friction in its bearings, which would also reduce the tension in the rope and thus the acceleration of the system.
These factors would need to be included in the model to more accurately predict the acceleration of the system.

</div>

+++

<br>


6. The diagram shows a particle P of mass $M \mathrm{~kg}$ suspended from two strings. The strings pass over smooth pulleys A and B and attach to hanging particles. The masses of particles below $A$ and $B$ are 2 kg and 3 kg respectively. The strings to $A$ and $B$ make angles of $30^{\circ}$ and $\alpha^{\circ}$ to the horizontal as shown in the diagram.
![](https://cdn.mathpix.com/cropped/2025_10_25_b19637d3b44bcb94728bg-1.jpg?height=261&width=680&top_left_y=2335&top_left_x=511)
(a) Calculate the value of $\alpha$.

+++ <span style="color:blue">Hint</span>

<div style="font-size: 22px;">

The horizontal components of the tensions applied on particle P must balance each other, and the force on the string is equal to the weight of the hanging particles. Use these relationships to set up equations and solve for \(\alpha\).

</div>

+++

+++ <span style="color:green">Solutions</span>

<div style="font-size: 22px">

To find the value of \(\alpha\), we need to analyze the forces acting on particle P and the hanging particles.
Let the tension in the string passing over pulley A be \(T_A\) and the tension in the string passing over pulley B be \(T_B\).
The weight of the hanging particles are:
- For the 2 kg particle: \(W_A = 2g = 2 \times 9.8 = 19.6 \, \text{N}\)
- For the 3 kg particle: \(W_B = 3g = 3 \times 9.8 = 29.4 \, \text{N}\)
Resolving the forces vertically and horizontally for particle P, we have:
1. Vertical components:
\[T_A \sin(30^\circ) + T_B \sin(\alpha) = Mg\]
2. Horizontal components:
\[T_A \cos(30^\circ) = T_B \cos(\alpha)\]
From the horizontal equation, we can express \(T_B\) in terms of \(T_A\):
\[T_B = T_A \frac{\cos(30^\circ)}{\cos(\alpha)}\]
Substituting this into the vertical equation:
\[T_A \sin(30^\circ) + \left(T_A \frac{\cos(30^\circ)}{\cos(\alpha)}\right) \sin(\alpha) = Mg\]
Using the values of \(\sin(30^\circ) = 0.5\) and \(\cos(30^\circ) = \sqrt{3}/2\):
\[T_A \left(0.5 + \frac{\sqrt{3}}{2} \tan(\alpha)\right) = Mg\]
To find \(\alpha\), we need to consider the tensions in the strings due to the hanging particles:
For pulley A:
\[T_A = W_A = 19.6 \, \text{N}\]
For pulley B:
\[T_B = W_B = 29.4 \, \text{N}\]
Substituting \(T_A\) and \(T_B\) into the horizontal equation:
\[19.6 \cos(30^\circ) = 29.4 \cos(\alpha)\]
Solving for \(\alpha\):
\[\cos(\alpha) = \frac{19.6 \cos(30^\circ)}{29.4} \approx 0.577\]
\[\alpha = \cos^{-1}(0.577) \approx 54.7^{\circ}.\]


</div>

+++

<br>


(b) Calculate the value of $M$.

+++ <span style="color:blue">Hint</span>

<div style="font-size: 22px;">

The vertical components of the tensions must sum to the weight of particle P. Use this relationship to set up an equation and solve for \(M\).

</div>

+++

+++ <span style="color:green">Solutions</span>

<div style="font-size: 22px">

Using the vertical components equation:
\[T_A \sin(30^\circ) + T_B \sin(\alpha) = Mg\]
Substituting \(T_A = 19.6 \, \text{N}\) and \(T_B = 29.4 \, \text{N}\):
\[19.6 \times 0.5 + 29.4 \times \sin(54.7^\circ) = Mg\]
Calculating the left side:
\[9.8 + 29.4 \times 0.819 = Mg\]
\[9.8 + 24.07 = Mg\]
\[Mg \approx 33.87 \, \text{N}\]
Solving for \(M\):
\[M = \frac{33.87}{9.8} \approx 3.46 \, \text{kg}.\]

</div>

+++

<br>


\((c)\) Calculate the magnitude of the total force exerted on the pulley at B.

+++ <span style="color:blue">Hint</span>

<div style="font-size: 22px;">

The total force exerted on pulley B is the vector sum of the tensions in the strings connected to it. The tensions are \(T_B\) and the weight of the hanging particle \(W_B\).

</div>
<p align="center">
<img src="/assets/Mech2_forces_Q6_c.jpg" alt="drawing" width="800"/>
</p>

</div>

+++

+++ <span style="color:green">Solutions</span>

<div style="font-size: 22px">

The total force exerted on pulley B is the vector sum of the tensions in the strings connected to it. The tensions are \(T_B\) and the weight of the hanging particle \(W_B\).

<p align="center">
<img src="/assets/Mech2_forces_Q6_c.jpg" alt="drawing" width="800"/>
</p>

The magnitude of the total force \(F_B\) can be calculated using the Pythagorean theorem:
\[F_B = \sqrt{(T_B \cdot \cos(\alpha))^2 + (T_B \cdot \sin (\alpha) + W_B)^2}\]
Substituting the values:
\[F_B = \sqrt{(29.4 \cdot \cos(54.7^\circ))^2 + (29.4 \cdot \sin (54.7^\circ) + 29.4)^2}\]
Calculating the components:
\[F_B = \sqrt{(29.4 \cdot 0.578)^2 + (29.4 \cdot 0.816 + 29.4)^2}\]
\[F_B = \sqrt{(16.9932)^2 + (24.07 + 29.4)^2}\]
\[F_B = \sqrt{288.769 + 2850.535} \approx \sqrt{3139.304} \approx 56.029 \, \text{N}.\]

</div>

+++

<br>


(d) Additional weights are attached to the particles below A and B. Explain what will happen to the particle P.


+++ <span style="color:green">Solutions</span>

<div style="font-size: 22px">

If additional weights are attached to the particles below pulleys A and B, the tensions in the strings will increase. This will result in a greater upward force acting on particle P.

</div>

+++

<br>


<br>

</div>
<p align="center">
<img src="/images/johnWick2.jpg" alt="drawing" width="500"/>
</p>

