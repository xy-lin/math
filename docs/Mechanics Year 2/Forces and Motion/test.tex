<p align="center">
<img src="/images/johnWick.jpg" alt="drawing" width="500"/>
</p>

<div style="font-size: 22px;">

<br><br>

1. A box of mass 4 kg rests on a rough plane inclined at $20^{\circ}$ to the horizontal.
(a) Calculate the magnitude of the normal reaction between the box and the plane.

<br>

+++ <span style="color:blue">Hint</span>

<hr style="border:1px solid red" >

<hr style="border:1px solid red" >

+++

<br>

+++ <span style="color:green">Solutions</span>

<hr style="border:1px solid red" >

<hr style="border:1px solid red" >

+++

<br>

(b) Calculate the magnitude of the frictional force between the box and the plane, indicatingthe direction clearly.
2. A particle P of mass 3 kg is suspended by two strings inclined at $60^{\circ}$ and $30^{\circ}$ to the horizontal as shown in the diagram. Calculate the tensions in the strings PA and PB.
![](https://cdn.mathpix.com/cropped/2025_10_25_b19637d3b44bcb94728bg-1.jpg?height=264&width=555&top_left_y=702&top_left_x=513)
3. A sledge of mass 5 kg is pulled over a frozen lake by a string at $55^{\circ}$ to the horizontal. The tension in the string is 20 N . There is a resistance to motion of 7 N . Calculate the acceleration of the sledge.
4. A lightbulb of mass 0.1 kg hangs from a wire. A horizontal force of $P \mathrm{~N}$ is applied to the lightbulb so that the angle between the wire and the vertical is $\alpha^{\circ}$.
(a) Draw a diagram showing all the forces on the lightbulb.
(b) In the case where $\alpha^{\circ}=30^{\circ}$, find the tension in the wire.
(c) Find the magnitude of $P$ in this situation.
(d) The force $P$ can be increased to a maximum value of 2 N , Find the maximum possible value for $\alpha$.
5. Two boxes of mass 20 kg and 25 kg are connected by a rope and are to be pulled up a smooth slope inclined at $15^{\circ}$ to the horizontal with the 20 kg box following the other. The 25 kg box is attached to a cable which is parallel to the plane and passes over a smooth pulley. The other end of a cable is attached to a sack of mass $M \mathrm{~kg}$ which hangs freely. The sack moves downwards with an acceleration of $0.14 \mathrm{~m} \mathrm{~s}^{-2}$.
(a) Draw a diagram to show all the forces acting on the boxes and the sack.
(b) Calculate the tension in the rope between the two boxes.
(c) Calculate the value of $M$.
(d) In practice the acceleration of the system with these masses is less than $0.14 \mathrm{~m} \mathrm{~s}^{-2}$. Explain how the modelling assumptions need to be changed to explain this.
6. The diagram shows a particle P of mass $M \mathrm{~kg}$ suspended from two strings. The strings pass over smooth pulleys A and B and attach to hanging particles. The masses of particles below $A$ and $B$ are 2 kg and 3 kg respectively. The strings to $A$ and $B$ make angles of $30^{\circ}$ and $\alpha^{\circ}$ to the horizontal as shown in the diagram.
![](https://cdn.mathpix.com/cropped/2025_10_25_b19637d3b44bcb94728bg-1.jpg?height=261&width=680&top_left_y=2335&top_left_x=511)
(a) Calculate the value of $\alpha$.
(b) Calculate the value of $M$.
(c) Calculate the magnitude of the total force exerted on the pulley at B .
(d) Additional weights are attached to the particles below A and B. Explain what will happen to the particle P.

<br>

</div>
<p align="center">
<img src="/images/johnWick2.jpg" alt="drawing" width="500"/>
</p>

