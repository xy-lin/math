<p align="center">
<img src="/images/topGun.webp" alt="drawing" width="500"/>
</p>

<div style="font-size: 22px;">

<br><br>

1. A rectangular lamina ABCD lies on a smooth surfance. The sides AB and AD measure 1.7 m and 1.4 m respectively. Horizontal forces of $20 \mathrm{~N}, 30 \mathrm{~N}$ and 25 N act at $\mathrm{D}, \mathrm{C}$ and the midpoint of BC at right angles to the edges as shown in the diagram.
![](https://cdn.mathpix.com/cropped/2025_10_25_9e8fbe5398c75afa7e82g-1.jpg?height=404&width=769&top_left_y=559&top_left_x=516)

Calculate the total moment of the three forces about A .

<br>

+++ <span style="color:blue">Hint</span>

<hr style="border:1px solid red" >

<hr style="border:1px solid red" >

+++

<br>

+++ <span style="color:green">Solutions</span>

<hr style="border:1px solid red" >

<hr style="border:1px solid red" >

+++

<br>

2. Matt is trying to keep a freely hinged door closed while Jill and Wini are on the other side trying to push it open. The door is 60 cm wide.
Matt can exert a maximum force of 225 N . Jill and Wini exert forces of 200 N and 150 N at $90^{\circ}$ to the door.
Jill exerts her force 10 cm from the edge of the door, and Wini in the middle as shown in the diagram.
![](https://cdn.mathpix.com/cropped/2025_10_25_9e8fbe5398c75afa7e82g-1.jpg?height=253&width=775&top_left_y=1501&top_left_x=519)

Determine whether Matt can keep the door closed.
3.
![](https://cdn.mathpix.com/cropped/2025_10_25_9e8fbe5398c75afa7e82g-1.jpg?height=295&width=912&top_left_y=1941&top_left_x=516)

A uniform rod AB of length 8 m and weight 180 N is held in horizontal equilibrium by two vertical wires. One wire is 1 m from A and the other 2 m from B .
(a) Draw a diagram showing all the forces acting on the rod.
(b) Calculate the tensions in the wires.

4. An injured climber is tied to a stretcher AB of length 2.5 m . The total mass of the climber and the stretcher is 90 kg .

In this question you should make the following modelling assumptions:
- the centre of mass, G , of the stretcher with climber is a distance 1.875 m from the end A of the stretcher, as shown in the diagram,
- all the forces acting on the system are in the same vertical plane.
![](https://cdn.mathpix.com/cropped/2025_10_25_9e8fbe5398c75afa7e82g-2.jpg?height=281&width=1024&top_left_y=699&top_left_x=513)

The lifting forces are each vertically upwards at the ends A and B of the stretcher, and the stretcher is held in horizontal equilibrium, as shown in the diagram above. Calculate the values of the lifting forces.
5.
![](https://cdn.mathpix.com/cropped/2025_10_25_9e8fbe5398c75afa7e82g-2.jpg?height=227&width=867&top_left_y=1353&top_left_x=516)

A uniform, horizontal, rigid shelf CD has a weight of 40 N and length 1.6 m . It is resting on two thin brackets A and B which are 0.4 m and 0.2 m respectively from C and D , as shown in the diagram above.
(a) Calculate the reaction forces of the brackets on the shelf.

An object is placed on the shelf so that its weight, $W \mathrm{~N}$, acts on the shelf at a distance $x \mathrm{~m}$ from C .
(b) Show that the vertical reaction force on the shelf at A is $(24-W(x-1.4)) \mathrm{N}$. Find a similar expression for the vertical reaction force on the shelf at B .
(c) For what values of $x$ will the shelf not tip up if $W=200$ ?

<br>

</div>
<p align="center">
<img src="/images/topGun2.webp" alt="drawing" width="500"/>
</p>
