<p align="center">
<img src="/images/topGun.webp" alt="drawing" width="500"/>
</p>

<div style="font-size: 22px;">

<br><br>

1. A rectangular lamina ABCD lies on a smooth surfance. The sides AB and AD measure 1.7 m and 1.4 m respectively. Horizontal forces of $20 \mathrm{~N}, 30 \mathrm{~N}$ and 25 N act at $\mathrm{D}, \mathrm{C}$ and the midpoint of BC at right angles to the edges as shown in the diagram.
![](https://cdn.mathpix.com/cropped/2025_10_25_9e8fbe5398c75afa7e82g-1.jpg?height=404&width=769&top_left_y=559&top_left_x=516)

Calculate the total moment of the three forces about A .

+++ <span style="color:green">Solutions</span>

<div style="font-size: 22px">

The moment of the $20 \mathrm{~N}$ force about A is $20 \times 0=0 \mathrm{Nm}$ (anticlockwise).
The moment of the $30 \mathrm{~N}$ force about A is $30 \times 1.7=51 \mathrm{Nm}$ (clockwise).
The moment of the $25 \mathrm{~N}$ force about A is $25 \times 0.7=17.5 \mathrm{Nm}$ (anticlockwise).
The total moment about A is $0+51-17.5 = 33.5\mathrm{Nm}$ (clockwise).

</div>

+++

<br>

2. Matt is trying to keep a freely hinged door closed while Jill and Wini are on the other side trying to push it open. The door is 60 cm wide.
Matt can exert a maximum force of 225 N . Jill and Wini exert forces of 200 N and 150 N at $90^{\circ}$ to the door.
Jill exerts her force 10 cm from the edge of the door, and Wini in the middle as shown in the diagram.
![](https://cdn.mathpix.com/cropped/2025_10_25_9e8fbe5398c75afa7e82g-1.jpg?height=253&width=775&top_left_y=1501&top_left_x=519)

Determine whether Matt can keep the door closed.

+++ <span style="color:blue">Hint</span>

<div style="font-size: 22px;">

To maximise his moment, Matt should push at the edge of the door opposite to the hinges.
Careful that the unit of length need to be consistent and better convert it to metres when calculating moments in Nm.

</div>

+++

+++ <span style="color:green">Solutions</span>

<div style="font-size: 22px">

The moment of Jill's force about the hinges is $200 \times 0.5=100 \mathrm{Nm}$ (anticlockwise).
The moment of Wini's force about the hinges is $150 \times 0.3=45 \mathrm{Nm}$ (anticlockwise).
The total moment about the hinges is $100+45=145 \mathrm{Nm}$ (anticlockwise).
The moment of Matt's force about the hinges is $225 \times 0.6=135 \mathrm{Nm}$ (clockwise).
Matt cannot keep the door closed as $135 \mathrm{Nm}<145 \mathrm{Nm}$.

</div>

+++

<br>

3.
![](https://cdn.mathpix.com/cropped/2025_10_25_9e8fbe5398c75afa7e82g-1.jpg?height=295&width=912&top_left_y=1941&top_left_x=516)

A uniform rod AB of length 8 m and weight 180 N is held in horizontal equilibrium by two vertical wires. One wire is 1 m from A and the other 2 m from B .
(a) Draw a diagram showing all the forces acting on the rod.


+++ <span style="color:green">Solutions</span>

<div style="font-size: 22px">


The forces acting on the rod are:
- The weight of the rod (180 N) acting vertically downwards at the centre of mass, which is 4 m from A .
- The tension in the wire 1 m from A , acting vertically upwards.
- The tension in the wire 2 m from B , acting vertically upwards.

</div>

+++

<br>

(b) Calculate the tensions in the wires.

+++ <span style="color:blue">Hint</span>

<div style="font-size: 22px;">

Can use different points to take moments about. Use centre of the rod, or use Point A, or Point B.
Remember to set up equations for vertical equilibrium as well.


</div>

+++

+++ <span style="color:green">Solutions</span>

<div style="font-size: 22px">

Let the tension in the wire 1 m from A be $T_{1} \mathrm{~N}$ and the tension in the wire 2 m from B be $T_{2} \mathrm{~N}$.
Taking moments about A :
\[180 \times 4=T_{2} \times 6+T_{1} \times 1\]
\[720=6 T_{2}+T_{1} \quad(1)\]
Also, since the rod is in vertical equilibrium:
\[T_{1}+T_{2}=180 \quad(2)\]
From (2), we have $T_{1}=180-T_{2}$.
Substituting into (1):
\[720=6 T_{2}+(180-T_{2})\]
\[720=5 T_{2}+180\]
\[540=5 T_{2}\]
\[T_{2}=108 \mathrm{~N}\]
Substituting back into (2):
\[T_{1}+108=180\]
\[T_{1}=72 \mathrm{~N}\]

<br>

=========
If use moments about center of rod:
Taking moments about the center of the rod:
\[T_{1} \times 3=T_{2} \times 2\]
\[3 T_{1}=2 T_{2} \quad(3)\]
Also, since the rod is in vertical equilibrium:
\[T_{1}+T_{2}=180 \quad(2)\]
From (3), we have $T_{2}=\frac{3}{2} T_{1}$.
Substituting into (2):
\[T_{1}+\frac{3}{2} T_{1}=180\]
\[\frac{5}{2} T_{1}=180\]
\[T_{1}=72 \mathrm{~N}\]
Substituting back into (3):
\[3 \times 72=2 T_{2}\]
\[T_{2}=108 \mathrm{~N}\]


</div>

+++

<br>

4. An injured climber is tied to a stretcher AB of length 2.5 m . The total mass of the climber and the stretcher is 90 kg.
In this question you should make the following modelling assumptions:
- the centre of mass, G , of the stretcher with climber is a distance 1.875 m from the end A of the stretcher, as shown in the diagram,
- all the forces acting on the system are in the same vertical plane.
![](https://cdn.mathpix.com/cropped/2025_10_25_9e8fbe5398c75afa7e82g-2.jpg?height=281&width=1024&top_left_y=699&top_left_x=513)

The lifting forces are each vertically upwards at the ends A and B of the stretcher, and the stretcher is held in horizontal equilibrium, as shown in the diagram above. Calculate the values of the lifting forces.

+++ <span style="color:blue">Hint</span>

<div style="font-size: 22px;">

Let the lifting force at A be $F_{A} \mathrm{~N}$ and the lifting force at B be $F_{B} \mathrm{~N}$.
The weight of the stretcher and climber is $90 \times 9.8 \mathrm{~N}$, acting vertically downwards at G, which is equal to sum of lifting forces at A and B.
Can use Point A or Point B to take moments about.

</div>

+++

+++ <span style="color:green">Solutions</span>

<div style="font-size: 22px">

Let the lifting force at A be $F_{A} \mathrm{~N}$ and the lifting force at B be $F_{B} \mathrm{~N}$.
The weight of the stretcher and climber is $90 \times 9.8=882 \mathrm{~N}$, acting vertically downwards at G .
Taking moments about A :
\[F_{B} \times 2.5=882 \times 1.875\]
\[2.5 F_{B}=1653.75\]
\[F_{B}=661.5 \mathrm{~N}\]
Also, since the stretcher is in vertical equilibrium:
\[F_{A}+F_{B}=882\]
Substituting the value of $F_{B}$ :
\[F_{A}+661.5=882\]
\[F_{A}=220.5 \mathrm{~N}\]

</div>

+++

<br>


5.
![](https://cdn.mathpix.com/cropped/2025_10_25_9e8fbe5398c75afa7e82g-2.jpg?height=227&width=867&top_left_y=1353&top_left_x=516)

A uniform, horizontal, rigid shelf CD has a weight of 40 N and length 1.6 m . It is resting on two thin brackets A and B which are 0.4 m and 0.2 m respectively from C and D , as shown in the diagram above.
(a) Calculate the reaction forces of the brackets on the shelf.

+++ <span style="color:blue">Hint</span>

<div style="font-size: 22px;">

Just use the principle of moments and vertical equilibrium.
Reaction forces at A and B are upwards, and is equal to the weight of the shelf downwards.

</div>

+++

+++ <span style="color:green">Solutions</span>

<div style="font-size: 22px">

Let the reaction force at A be $R_{A} \mathrm{~N}$ and the reaction force at B be $R_{B} \mathrm{~N}$.
Taking moments about C :
\[40 \times 0.8=R_{B} \times 1.4+R_{A} \times 0.4\]
\[32=1.4 R_{B}+0.4 R_{A} \quad(1)\]
Also, since the shelf is in vertical equilibrium:
\[R_{A}+R_{B}=40 \quad(2)\]
From (2), we have $R_{B}=40-R_{A}$.
Substituting into (1):
\[32=1.4(40-R_{A})+0.4 R_{A}\]
\[32=56-1.4 R_{A}+0.4 R_{A}\]
\[R_{A}=24 \mathrm{~N}\]
Substituting back into (2):
\[R_{B}=16 \mathrm{~N}\]

</div>

+++

<br>

An object is placed on the shelf so that its weight, $W \mathrm{~N}$, acts on the shelf at a distance $x \mathrm{~m}$ from C .
(b) Show that the vertical reaction force on the shelf at A is $(24-W(x-1.4)) \mathrm{N}$. Find a similar expression for the vertical reaction force on the shelf at B.

+++ <span style="color:blue">Hint</span>

<div style="font-size: 22px;">

Similar to part (a), but now there is an additional weight W acting downwards at a distance x from C.

</div>

+++

+++ <span style="color:green">Solutions</span>

<div style="font-size: 22px">

Let the reaction force at A be $R_{A} \mathrm{~N}$ and the reaction force at B be $R_{B} \mathrm{~N}$.
Taking moments about C :
\[40 \times 0.8+W \times x=R_{B} \times 1.4+R_{A} \times 0.4\]
\[32+W x=1.4 R_{B}+0.4 R_{A} \quad(1)\]
Also, since the shelf is in vertical equilibrium:
\[R_{A}+R_{B}=40+W \quad(2)\]
From (2), we have $R_{B}=40+W-R_{A}$.
Substituting into (1):
\[32+W x=1.4(40+W-R_{A})+0.4 R_{A}\]
\[32+W x=56+1.4 W-1.4 R_{A}+0.4 R_{A}\]
\[-24+W(x-1.4)=-R_{A}\]
\[R_{A}=24-W(x-1.4) \mathrm{~N}\]
Substituting back into (2):
\[R_{B}=16+W(x-1.4) \mathrm{~N}\]

</div>

+++

<br>

\((c)\) For what values of $x$ will the shelf not tip up if $W=200$ ?

+++ <span style="color:blue">Hint</span>

<div style="font-size: 22px;">

Now need to use Point A as the pivot point to take moments about.
And remember, when the shelf is just about to tip, the reaction force at B is zero, so we just need to consider the moments from weight of shelf and the object.
</div>

+++

+++ <span style="color:green">Solutions</span>

<div style="font-size: 22px">

Taking moments about A when the shelf is just about to tip:
\[40 \times 0.4 = 200 \times(0.4-x)\]
\[16 = -200 x+80\]
\[200 x=64\]
\[x=0.32 \mathrm{~m}\]


</div>

+++

<br>

</div>
<p align="center">
<img src="/images/topGun2.webp" alt="drawing" width="500"/>
</p>
