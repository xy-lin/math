<p align="center">
<img src="/images/deadPoet.jpg" alt="drawing" width="500"/>
</p>

<div style="font-size: 22px;">

<br><br>

1. A box is at rest on a rough horizontal floor. A horizontal force of 400 N acts on the box and the box is on the point of moving. The coefficient of friction between the box and the floor is 0.65 . Calculate the weight of the box.

<br>

+++ <span style="color:blue">Hint</span>

<hr style="border:1px solid red" >

<hr style="border:1px solid red" >

+++

<br>

+++ <span style="color:green">Solutions</span>

<hr style="border:1px solid red" >

<hr style="border:1px solid red" >

+++

<br>

2. In the system below, the coefficient of friction between the block and the table is 0.2 .
![](https://cdn.mathpix.com/cropped/2025_10_25_f05d4d9c1baeb4d61aacg-1.jpg?height=432&width=1131&top_left_y=608&top_left_x=391)
Find the acceleration of the system and the tensions $T_{1}$ and $T_{2}$.
3. A small object of mass 4 kg is placed on a plane inclined at $25^{\circ}$ to the horizontal. It is held in place by a light inextensible string that is parallel to the plane, passes over a smooth pulley and is attached to a sphere of mass $m \mathrm{~kg}$ which hangs freely as shown in the diagram.
![](https://cdn.mathpix.com/cropped/2025_10_25_f05d4d9c1baeb4d61aacg-1.jpg?height=235&width=926&top_left_y=1410&top_left_x=394)
(a) Rishi models the system with the modelling assumption that the plane is smooth. Calculate the value of $m$ for the system to be in equilibrium.
(b) Rishi finds that the system is in equilibrium when the mass of the sphere is 2.8 kg . He improves his model by modelling the plane as rough with coefficient of friction $\mu$ between the object and the plane. Find the range of values in which $\mu$ must lie. [5]
4. The diagram shows a mass of 50 kg on a slope which makes an angle of $30^{\circ}$ with the horizontal. The coefficient of friction between the mass and the slope is 0.25 . You may assume that the mass does not tip up.
![](https://cdn.mathpix.com/cropped/2025_10_25_f05d4d9c1baeb4d61aacg-1.jpg?height=335&width=472&top_left_y=2227&top_left_x=491)
Find the magnitude of the force $T$ if
(a) the mass is about to move down the slope
(b) the mass is accelerating at $5 \mathrm{~ms}^{-2}$ up the slope.
5. George pulls a crate of mass $M \mathrm{~kg}$ across a rough horizontal floor using a light inextensigle string which makes an angle $\theta$ to the horizontal. The coefficient of fricton between the floor and the box is $\mu$. The acceleration of the box across the floor is $a \mathrm{~ms}^{-2}$. George models the situation assuming that there are no other resistive forces.
(a) Show that $T$, the tension in the string is given by $T=\frac{M(a+\mu g)}{(\cos \theta+\mu \sin \theta)}$.
(b) George knows that when $M=40, \theta=20$ and $T=220$, the box is on the point of moving. Use the expression from part (a) to calculate a value of $\mu$.
6. Two small blocks of mass 0.8 kg and 1.2 kg are attached with a light inextensible rod which is parallel to the bases of the blocks. The blocks are placed on a plane inclined at $30^{\circ}$ to the horizontal with the 1.2 kg block higher than the 0.8 kg block. The coefficient of friction between the 0.8 kg mass and the plane is 0.5 and that between the 1.2 kg mass and the plane is 0.3 .
![](https://cdn.mathpix.com/cropped/2025_10_25_f05d4d9c1baeb4d61aacg-2.jpg?height=233&width=861&top_left_y=1267&top_left_x=502)
(a) Determine the acceleration of the system.
(b) Determine the magnitude of the force in the rod making clear whether it is in tension or compression.

<br>

</div>
<p align="center">
<img src="/images/deadPoet2.jpg" alt="drawing" width="500"/>
</p>
